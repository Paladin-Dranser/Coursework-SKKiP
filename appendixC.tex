\sectionWithoutNumber{Дадатак В}
\vspace{-0.8\baselineskip}
\begin{center}
    \textbf{Выбар канала сувязі ў залежнасці ад прапускной здольнасці}
\end{center}

\vspace{-0.5\baselineskip}
\begin{table}[h!]
    \renewcommand{\thetable}{В.1}
    \caption{Выбар канала сувязі ў залежнасці ад
             прапускной здольнасці}
    \begin{tabularx}{\textwidth}{|c|>{\centering\arraybackslash}X|c|}
        \hline
        Маршрутызатар & \makecell[c]{Сумарная прапускная здольнасць, Мбіт/с}
        & Канал сувязі \\
        \hline
        M1 & 3649,94 & 10GE \\
        \hline
        M2 & 3649, 94 & 10GE \\
        \hline
        M3 & 3652,85 & 10GE \\
        \hline
        M4 & 3654,16 & 10GE \\
        \hline
        M5 & 3653,25 & 10GE \\
        \hline
        M6 & 3655,36 & 10GE \\
        \hline
        M7 & 3654,16 & 10GE \\
        \hline
        M8 & 3653,52 & 10GE \\
        \hline
        M9 & 3653,39 & 10GE \\
        \hline
        M10 & 3652,85 & 10GE \\
        \hline
    \end{tabularx}
    \label{table: channels of communication}
\end{table}
