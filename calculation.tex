\section{Разлік сеткавых параметраў праектуемай сеткі}

\subsection{Матэматычная мадэль разліку сеткавых параметраў}

\subsection{Разлік матрыцы інфармацыйнага прыцягнення}

Для разліку матрыцы інфармацыйнага прыцягнення скарыстаемся
праграмай генерацыі матрыцы прыцягнення (для сетак пакетнай камутацыі).

У аснове праграмы закладзены наступны алгарытм:
\begin{enumerate}
    \item задаецца колькасць камутатараў;
    \item задаецца колькасць IP-абанентаў для зоны абслугоўвання
          кожнага маршрутызатара;
    \item задаецца ўдзельная абаненцкая нагрузка,
          абаненцкая нагрузка і іншыя параметры;
    \item памнажэннем колькасці IP-абанентаў (па ўсіх зонах)
          на ўдзельную абаненцкую нагрузку знаходзіцца
          сумарны знешні трафік (у Эрлангах);
    \item атрыманыя хвіліна-займанні пералічваем у пакетызаваны
          трафік (пакетаў/с).
          Для гэтага памнажаецца сумарны трафік у Эрлангах на
          хуткасць працы кодэка МПУ (мовапераўтваральнага ўстройства),
          паўзы пры гэтым выключаюцца,
          г. зн. іх не кадзіруем. Дзяленнем выніку на
          аб'ём пакета знаходзіцца інтэнсіўнасць
          сумарнага ўваходнага патоку (пакетаў/с);
    \item расшчэпліваецца сумарны ўваходны паток (пакетаў/с) па
          кірунках сувязі з улікам нераўнамернасці размеркавання
          абанентаў па зонах абслугоўвання маршрутызатараў;
    \item вынікам працы з'яўляецца матрыцы інфармацыйнага прыцягнення ---
          квадратная таб\-лі\-ца размернасці NxN (N - зыходная колькасць
          маршрутызатараў).
\end{enumerate}

У табліцы \ref{table: The number of subscribers} прадстаўлена
колькасць абанентаў на кожны маршрутызатар.

\begin{table}[!h]
    \caption{Колькасць абанентаў на маршрутызатары}
    \begin{tabularx}{\textwidth}{ | >{\centering\arraybackslash}X
                                  | >{\centering\arraybackslash}X
                                  | >{\centering\arraybackslash}X | }
    \hline
        Нумар маршрутызатара & Колькасць абанентаў & Усяго абанентаў \\
    \hline
        1 & 1200
        &
        \multirow{10}{*}{33200} \\
    \cline{1-2}
        2 & 1200 & \\
    \cline{1-2}
        3 & 3200 & \\
    \cline{1-2}
        4 & 4200 & \\
    \cline{1-2}
        5 & 3500 & \\
    \cline{1-2}
        6 & 5200 & \\
    \cline{1-2}
        7 & 4200 & \\
    \cline{1-2}
        8 & 3700 & \\
    \cline{1-2}
        9 & 3600 & \\
    \cline{1-2}
        10 & 3200 & \\
    \hline
    \end{tabularx}
    \label{table: The number of subscribers}
\end{table}

Зыходныя і прамежкавыя даныя, якія будуць выкарыстоўвацца для
разліку параметраў сеткі, прыведзеныя ў табліцы
\ref{table: Original data}

\clearpage

\begin{table}[!h]
    \caption{Зыходныя і прамежкавыя даныя}
    \begin{tabularx}{\textwidth}{ | p{0.36\textwidth}
                                  | >{\centering\arraybackslash}X
                                  | >{\centering\arraybackslash}X | }
        \hline
            \multicolumn{1}{|c|}{Найменне паказальніка}
            &
            Адзінка вымярэння
            &
            Значэнне паказальніка \\
        \hline
            Колькасць маршрутызатараў & & 10 \\
        \hline
            Сярэдняя працягласць размовы & секунд & 100 \\
        \hline
            \makecell[l]{Інтэнсіўнасць выклікаў (у гадзіну\\ ад абанента)}
            & &
            2 \\
        \hline
            \makecell[l]{Удзельная выходная абаненцкая \\ нагрузка}
            & Эрлангаў & 0,0556 \\
        \hline
            \makecell[l]{Сумарная ўваходная абаненцкая \\ нагрузка}
            & Эрлангаў & 1826 \\
        \hline
            Аб'ём пакета & байт & 592 \\
        \hline
            Хуткасць працы МПУ & біт/с & 16000 \\
        \hline
            Сярэдняя працягласць фанемы & секунд & 1,34 \\
        \hline
            Сярэдняя працягласць паўзы & секунд & 1,67 \\
        \hline
            \makecell[l]{Сярэдняя колькасць актыўных \\ перыядаў у размове}
            & &
            16 \\
        \hline
            Сумарны знешні трафік & пакетаў/с & 9411,9 \\
        \hline
    \end{tabularx}
    \label{table: Original data}
\end{table}

Матрыца інфармацыйнага прыцягнення, якая была атрыманая
пры дапамозе праграмы генерацыі матрыцы прыцягнення,
прадстаўленая ў табліцы \ref{table: Matrix}.

\begin{table}[!h]
    \caption{Матрыца інфармацыйнага прыцягнення}
    \begin{tabularx}{\textwidth}{ |>{\centering\arraybackslash}X
                                  |c|c|c|c|c|c|c|c|c|c|}
        \hline
        \makecell{Нумар\\ маршрутызатара}
            & M1   & M2   & M3    & M4    & M5    & M6    & M7    & M8    & M9    & M10 \\
        \hline
        M1  & 12,3 & 12,3 & 32,8  & 43,0  & 35,9  & 53,3  & 43,0  & 37,9  & 36,9  & 32,8 \\
        \hline
        M2  & 12,3 & 12,3 & 32,8  & 43,0  & 35,9  & 53,3  & 43,0  & 37,9  & 36,9  & 32,8 \\
        \hline
        M3  & 32,8 & 32,8 & 87,4  & 114,8 & 95,6  & 142,1 & 114,8 & 101,1 & 98,4  & 87,4 \\
        \hline
        M4  & 43,0 & 43,0 & 114,8 & 150,6 & 125,5 & 186,5 & 150,6 & 132,7 & 129,1 & 114,8 \\
        \hline
        M5  & 35,9 & 35,9 & 95,6  & 125,5 & 104,4 & 155,4 & 125,5 & 110,6 & 107,6 & 95,6 \\
        \hline
        M6  & 53,3 & 53,3 & 142,1 & 186,5 & 156,4 & 230,9 & 186,5 & 164,3 & 159,8 & 142,1 \\
        \hline
        M7  & 43,0 & 43,0 & 114,8 & 150,6 & 125,5 & 186,5 & 150,6 & 132,7 & 129,1 & 114,8 \\
        \hline
        M8  & 37,9 & 37,9 & 101,1 & 132,7 & 110,6 & 164,3 & 132,7 & 116,9 & 113,7 & 101,1 \\
        \hline
        M9  & 36,9 & 36,9 & 98,4  & 129,1 & 107,6 & 159,8 & 129,1 & 113,7 & 110,7 & 98,4 \\
        \hline
        M10 & 32,8 & 32,8 & 87,4  & 114,8 & 95,6  & 142,1 & 114,8 & 101,1 & 98,4  & 87,4 \\
        \hline
    \end{tabularx}
    \label{table: Matrix}
\end{table}

Калі прасуміраваць ўсе значэнні з матрыцы інфармацыйнага прыцягнення,
атрымаем сумарны знешні трафік роўны 9411,9 пакетаў/с,
што адпавядае зыходным даным з табліцы
\ref{table: Original data}.

\subsection{Разлік канальнага рэсурсу праектуемай сеткі}
