\section{Разлік сеткавых параметраў праектуемай сеткі}

\subsection{Матэматычная мадэль разліку сеткавых параметраў}

Праектаванне IP-сеткі ёсць разлік асноўных характарыстык элементаў сеткі.
Да гэтых характарыстык адносяцца:
\begin{enumerate}
    \item загрузка $i$-га канала;
    \item затрымка пакета ў канале сувязі (канальная затрымка);
    \item сярэднясеткавая затрымка;
    \item скразная затрымка;
    \item імавернасць своечасовай дастаўкі пакета.
\end{enumerate}

Загрузка $i$-га канала характаразуецца ступенню загружанасці
канала сувязі і роўная:
\begin{equation}
    \rho_i = \frac{\lambda_i}{\mu_i}, i = 1,2,3 \ldots m,
\end{equation}
\begin{Explanation}
    \item[дзе] $\rho_i$ -- загрузка $i$-га канала;
    \item $\lambda_i$ -- інтэнсіўнасць уваходнага патоку ў
                         $i$-ы канал, пакетаў/с
    \item $\mu_i$ -- інтэнсіўнасць абслугоўвання $i$-м
                     каналам, пакетаў/с.
\end{Explanation}

Інтэнсіўнасць уваходнага патоку ў $i$-ы канал знаходзіцца з
матрыцы інфармацыйнага прыцягнення [$\gamma_{jk}$] і роўная:
\begin{equation}
    \lambda_i = \sum_{jk \in i} \gamma_{jk},
\end{equation}
\begin{Explanation}
    \item[дзе] $\gamma_{jk}$ -- інтэнсіўнасць патоку, што
               падлягае перадачы паміж $j$-м і $k$-м маршрутызатарам
               (элемент матрыцы інфармацыйнага прыцягнення);
\end{Explanation}

Пры разліку канальнага трафіка $\lambda_i$ суміруюцца толькі тыя патокі
$\gamma_{jk}$, якія праходзяць па $i$-му каналу (пакетаў/с).
Маршруты паміж усімі парамі $j$-$k$ вызначаюцца пры дапамозе
алгарытмаў пошуку найкарацейшых шляхоў, напрыклад, пры дапамозе
алгарытма Флойда.

Інтэнсіўнасць абслугоўвання $i$-м каналам знаходзіцца па формуле:
\begin{equation}
    \mu_i = \frac{C_i}{V},
\end{equation}
\begin{Explanation}
    \item[дзе] $C_i$ -- прапускная здольнасць канала, біт/с;
    \item $V$ -- памер пакета, біт
\end{Explanation}

Адной з самых важных характарыстык, якая адносіцца да паказальнікаў
якасці абслугоўвання, з'яўляецца затрымка пакета, якая
вылічваецца для асобных каналаў, маршрутаў і агулам па сетцы.

Затрымка пакета ў канале сувязі (канальная затрымка):
\begin{equation}
    T_i = \frac{1}{\mu_i - \lambda_i},
\end{equation}
\begin{Explanation}
    \item[дзе] $T_i$ -- затрымка ў $i$-м канале сувязі, с
\end{Explanation}
Сярэднясеткавая затрымка пакета:
\begin{equation}
    \Myoverline{T} =
        \frac{1}{\Lambda}\sum_{i = 1}^{m} {\lambda_i \cdot T_i} =
        \sum_{i = 1}^{m} {\alpha_i \cdot T_i},
        \alpha_i = \frac{\lambda_i}{\Lambda},
\end{equation}
\begin{Explanation}
    \item[дзе] $\Lambda$ -- сумарны знешні трафік
                            (сума ўсіх элементаў матрыцы
                            прыцягнення), пакетаў/с;
    \item $\alpha_i$ -- дапаможны вагавы каэфіцыент для
                        $i$-га канала, які паказвае
                        <<ўклад>> $i$-га канала ў
                        сярэднясеткавую затрымку
\end{Explanation}

Сярэднясеткавая затрымка карысна для параўнання розных праектаў.

Скразная затрымка (затрымка <<з канца ў канец>> альбо end-to-end)
$T_{\text{end-to-end}}$ на асобным маршруце ўяўляе сабой суму
канальных затрымак $T_i$ усіх каналаў, якія ўваходзяць у маршрут,
які разглядаецца, а таксама сярэдні час апрацоўкі пакета ў канцавых
$T_{\text{SP}}$ і транзітных $T_{\text{STP}}$ маршрутызатарах:
\begin{equation}
    T_{\text{end-to-end}} = 2 \cdot T_{\text{SP}} + \pi \cdot T_{\text{STP}}
                            + \sum_{i = 1}^{\pi + 1} {T_i},
\end{equation}
\begin{Explanation}
    \item[дзе] $\pi$ -- колькасць транзітных маршрутызатараў, якія
                        ўваходзяць у злучэнне, якое разглядаецца
\end{Explanation}

Для забеспячэння зададзенай якасці абслугоўвання скразная затрымка
моўнага пакета для любога маршрута не павінна быць больш за 0,15 секунд.

Імавернасць своечасовай дастаўкі пакета --- гэта імавернасць таго, што
моўны пакет будзе дастаўлены да атрымальніка за час, які не перавышае
зададзены (дапушчальны) час, для дадзенага тыпу трафіка:

\begin{equation}
    P\{ \Myoverline{T} \leq t_{\text{з}} \} =
                       1 - e^{\frac{-t_{\text{з}}}{\Myoverline{T}}},
\end{equation}
\begin{Explanation}
    \item[дзе] $t_{\text{з}}$ -- зададзены час дастаўкі пакета праз
                                 усю сетку,
    \item $\Myoverline{T}$ -- канальная затрымка (затрымка
                              на маршруце) альбо сярэднясеткавая затрымка
\end{Explanation}

Лічыцца, што велічыня, якая дапаўняе $P\{.\}$ да 1, гэта страты, якія
для VoIP не павінны быць больш за 3\%.

\subsection{Разлік матрыцы інфармацыйнага прыцягнення}

Для разліку матрыцы інфармацыйнага прыцягнення скарыстаемся
праграмай генерацыі матрыцы прыцягнення (для сетак пакетнай камутацыі).

У аснове праграмы закладзены наступны алгарытм:
\begin{enumerate}
    \item задаецца колькасць камутатараў;
    \item задаецца колькасць IP-абанентаў для зоны абслугоўвання
          кожнага маршрутызатара;
    \item задаецца ўдзельная абаненцкая нагрузка,
          абаненцкая нагрузка і іншыя параметры;
    \item памнажэннем колькасці IP-абанентаў (па ўсіх зонах)
          на ўдзельную абаненцкую нагрузку знаходзіцца
          сумарны знешні трафік (у Эрлангах);
    \item атрыманыя хвіліна-займанні пералічваем у пакетызаваны
          трафік (пакетаў/с).
          Для гэтага памнажаецца сумарны трафік у Эрлангах на
          хуткасць працы кодэка МПУ (мовапераўтваральнага ўстройства),
          паўзы пры гэтым выключаюцца,
          г. зн. іх не кадзіруем. Дзяленнем выніку на
          аб'ём пакета знаходзіцца інтэнсіўнасць
          сумарнага ўваходнага патоку (пакетаў/с);
    \item расшчэпліваецца сумарны ўваходны паток (пакетаў/с) па
          кірунках сувязі з улікам нераўнамернасці размеркавання
          абанентаў па зонах абслугоўвання маршрутызатараў.
          Расшчэпліванне патоку <<B>> выконваецца ў адпаведнасці
          з вядомым метадам <<Equal Distribution>>, згодна з якім
          элемент матрыцы <<B[i,j]>> вызначаюцца перамножваннем
          абаненцкіх ёмістасцей маршрутызатараў, якія разглядаем,
          на сумарны трафік <<B>> і дзяленнем рэзультату на
          сумарную абаненцкую ёмістасць сеткі ў квадраце:
          $TR(I,J) = EMK(I) \cdot EMK(J) \cdot B / ES / ES.$
    \item вынікам працы з'яўляецца матрыцы інфармацыйнага прыцягнення ---
          квадратная таб\-лі\-ца размернасці NxN (N -- зыходная колькасць
          маршрутызатараў).
\end{enumerate}

У табліцы \ref{table: The number of subscribers} прадстаўлена
колькасць абанентаў на кожны маршрутызатар.

\begin{table}[htp]
    \caption{Колькасць абанентаў на маршрутызатары}
    \begin{tabularx}{\textwidth}{ | >{\centering\arraybackslash}X
                                  | >{\centering\arraybackslash}X
                                  | >{\centering\arraybackslash}X | }
    \hline
        Нумар маршрутызатара & Колькасць абанентаў & Усяго абанентаў \\
    \hline
        1 & 1200
        &
        \multirow{10}{*}{33200} \\
    \cline{1-2}
        2 & 1200 & \\
    \cline{1-2}
        3 & 3200 & \\
    \cline{1-2}
        4 & 4200 & \\
    \cline{1-2}
        5 & 3500 & \\
    \cline{1-2}
        6 & 5200 & \\
    \cline{1-2}
        7 & 4200 & \\
    \cline{1-2}
        8 & 3700 & \\
    \cline{1-2}
        9 & 3600 & \\
    \cline{1-2}
        10 & 3200 & \\
    \hline
    \end{tabularx}
    \label{table: The number of subscribers}
\end{table}

\newpage

Зыходныя і прамежкавыя даныя, якія будуць выкарыстоўвацца для
разліку параметраў сеткі, прыведзеныя ў табліцы
\ref{table: Original data}

\begin{table}[htp]
    \caption{Зыходныя і прамежкавыя даныя}
    \begin{tabularx}{\textwidth}{ | p{0.36\textwidth}
                                  | >{\centering\arraybackslash}X
                                  | >{\centering\arraybackslash}X | }
        \hline
            \multicolumn{1}{|c|}{Найменне паказальніка}
            &
            Адзінка вымярэння
            &
            Значэнне паказальніка \\
        \hline
            Колькасць маршрутызатараў & & 10 \\
        \hline
            Сярэдняя працягласць размовы & секунд & 100 \\
        \hline
            \makecell[l]{Інтэнсіўнасць выклікаў (у гадзіну\\ ад абанента)}
            & &
            2 \\
        \hline
            \makecell[l]{Удзельная выходная абаненцкая \\ нагрузка}
            & Эрлангаў & 0,0556 \\
        \hline
            \makecell[l]{Сумарная ўваходная абаненцкая \\ нагрузка}
            & Эрлангаў & 1826 \\
        \hline
            Аб'ём пакета & байт & 592 \\
        \hline
            Хуткасць працы МПУ & біт/с & 16000 \\
        \hline
            Сярэдняя працягласць фанемы & секунд & 1,34 \\
        \hline
            Сярэдняя працягласць паўзы & секунд & 1,67 \\
        \hline
            \makecell[l]{Сярэдняя колькасць актыўных \\ перыядаў у размове}
            & &
            16 \\
        \hline
            Сумарны знешні трафік & пакетаў/с & 9411,9 \\
        \hline
    \end{tabularx}
    \label{table: Original data}
\end{table}

Матрыца інфармацыйнага прыцягнення, якая была атрыманая
пры дапамозе праграмы генерацыі матрыцы прыцягнення,
прадстаўленая ў табліцы \ref{table: Matrix}.

\begin{table}[htp]
    \caption{Матрыца інфармацыйнага прыцягнення}
    \begin{tabularx}{\textwidth}{|>{\centering\arraybackslash}m{1.65cm}
                                 |c|c|c|c|c|c|c|c|c|c|}
        \hline
        Mаршрутызатар
            & M1   & M2   & M3    & M4    & M5    & M6    & M7    & M8    & M9    & M10 \\
        \hline
        M1  & 12,3 & 12,3 & 32,8  & 43,0  & 35,9  & 53,3  & 43,0  & 37,9  & 36,9  & 32,8 \\
        \hline
        M2  & 12,3 & 12,3 & 32,8  & 43,0  & 35,9  & 53,3  & 43,0  & 37,9  & 36,9  & 32,8 \\
        \hline
        M3  & 32,8 & 32,8 & 87,4  & 114,8 & 95,6  & 142,1 & 114,8 & 101,1 & 98,4  & 87,4 \\
        \hline
        M4  & 43,0 & 43,0 & 114,8 & 150,6 & 125,5 & 186,5 & 150,6 & 132,7 & 129,1 & 114,8 \\
        \hline
        M5  & 35,9 & 35,9 & 95,6  & 125,5 & 104,4 & 155,4 & 125,5 & 110,6 & 107,6 & 95,6 \\
        \hline
        M6  & 53,3 & 53,3 & 142,1 & 186,5 & 155,4 & 230,9 & 186,5 & 164,3 & 159,8 & 142,1 \\
        \hline
        M7  & 43,0 & 43,0 & 114,8 & 150,6 & 125,5 & 186,5 & 150,6 & 132,7 & 129,1 & 114,8 \\
        \hline
        M8  & 37,9 & 37,9 & 101,1 & 132,7 & 110,6 & 164,3 & 132,7 & 116,9 & 113,7 & 101,1 \\
        \hline
        M9  & 36,9 & 36,9 & 98,4  & 129,1 & 107,6 & 159,8 & 129,1 & 113,7 & 110,7 & 98,4 \\
        \hline
        M10 & 32,8 & 32,8 & 87,4  & 114,8 & 95,6  & 142,1 & 114,8 & 101,1 & 98,4  & 87,4 \\
        \hline
        Усяго & 340,2 & 340,2 & 907,2 & 1190,6 & 992,2 & 1474,2 & 1190,6 & 1048,9 & 1020,6 & 907,2 \\
        \hline
    \end{tabularx}
    \label{table: Matrix}
\end{table}

Калі прасуміраваць ўсе значэнні з матрыцы інфармацыйнага прыцягнення,
атрымаем сумарны знешні трафік роўны 9411,9 пакетаў/с,
што адпавядае зыходным даным з табліцы
\ref{table: Original data} і з'яўляецца аптымальнай матрыцай
інфармацыйнага прыцягнення з прыведзеных у выніках праграмы.

\subsection{Разлік канальнага рэсурсу праектуемай сеткі}

Для разліку канальнага рэсурсу праектуемай сеткі скарыстаемся
праграмай <<Разлік прапускной здольнасці каналаў сувязі>>.

Дадзеная праграма выконвае падлік параметраў па наступным
алгарытме:
\begin{enumerate}
    \item задаёцца колькасць абанентаў і хуткасць працы кодэка.
          Іншыя параметры задаюцца аналагічна з вышэй апісанай
          праграмай і вылічваюцца аўтаматычна;
    \item адбываецца памнажэнне колькасці абанентаў на ўдзельную
          абаненцкую нагрузку і хут\-касць працы кодэка, а затым
          дзяленнем на аб'ём пакета знаходзіцца інтэнсіўнасць
          уваходнага патоку (пакетаў/с).
\end{enumerate}

Значэнне сеткавых параметраў атрымліваем пасля ўводу ў праграму
інтэнсіўнасці уваходнай нагрузкі (з МІП) для кожнага маршрутызатара
ў напрамку іншых маршрутызатараў.

Разлічым сеткавыя параметры для карыстальнікаў VoIP. Пры гэтым пад
увагу прымаюцца зададзеныя нормы для добрай якасці дастаўкі:
\begin{enumerate}
    \item затрымка пакета не павінна перавышаць 10 мс;
    \item імавернасць страты пакета не павінна перавышаць 3\%.
\end{enumerate}

Атрыманыя рэзультаты аформім у выглядзе табліцы
\ref{table: VoIP Parameters}, дзе:
\begin{enumerate}
    \item $C$ -- хуткасць (прапускная здольнасць), біт/с;
    \item $\rho$ -- загрузка;
    \item $t_{\text{зат}}$ -- затрымка, c;
    \item $P_{\text{сд}}$ -- імавернасць своечасовай дастаўкі;
    \item $P_{\text{с}}$ -- імавернасць страты.
\end{enumerate}

\setlength\LTleft{0pt}
\setlength\LTright{0pt}
\fontsize{10}{12}\selectfont
\setlength\extrarowheight{0pt}
\begin{longtable}{|>{\centering\arraybackslash}m{0.195\textwidth}
                  |>{\centering\arraybackslash}m{0.15\textwidth}
                  |>{\centering\arraybackslash}m{0.05\textwidth}
                  |>{\centering\arraybackslash}m{0.15\textwidth}
                  |>{\centering\arraybackslash}m{0.15\textwidth}
                  |>{\centering\arraybackslash}m{0.15\textwidth}|}
    \caption{Сеткавыя параметры для маршрутызатараў
             на ўсе кірункі сувязі} \\

    \hline
    Накірунак сувязі & $C$, біт/с & $\rho$ & $t_{\text{зат}}$, c
    & $P_{\text{сд}}$ & $P_{\text{c}}$ \\
    \hline
    1 & 2 & 3 & 4 & 5 & 6 \\
    \hline
    \endfirsthead

        \multicolumn{6}{l}{\hspace{-0.2cm}\normalsize Працяг табліцы \thetable} \\
        \hline
        1 & 2 & 3 & 4 & 5 & 6 \\
        \hline
    \endhead

    \multirow{4}{*}{M1-M2}
     &	291264	&	0,2	&	0,020325	&	0,956054	&	0,043946 \\ \cline{2-6}
     &	145632	&	0,4	&	0,054201	&	0,824214	&	0,175786 \\ \cline{2-6}
     &	97088	&	0,6	&	0,121951	&	0,604483	&	0,395517 \\ \cline{2-6}
     &	72816	&	0,8	&	0,325203	&	0,296858	&	0,703142 \\ \hline
    \multirow{4}{*}{M1-M3}
    &	776704	&	0,2	&	0,007622	&	0,98352	&	0,01648	\\ \cline{2-6}
    &	388352	&	0,4	&	0,020325	&	0,93408	&	0,06592	\\ \cline{2-6}
    &	258901,33	&	0,6	&	0,045732	&	0,851681	&	0,148319	\\ \cline{2-6}
    &	194176	&	0,8	&	0,121951	&	0,736322	&	0,263678	\\ \hline
    \multirow{4}{*}{M1-M4}
    &	1018240	&	0,2	&	0,005814	&	0,987429	&	0,012571	\\ \cline{2-6}
    &	509120	&	0,4	&	0,015504	&	0,949717	&	0,050283	\\ \cline{2-6}
    &	339413,33	&	0,6	&	0,034884	&	0,886864	&	0,113136	\\ \cline{2-6}
    &	254560	&	0,8	&	0,093023	&	0,798869	&	0,201131	\\ \hline
    \multirow{4}{*}{M1-M5}
    &	850112	&	0,2	&	0,006964	&	0,984943	&	0,015057	\\ \cline{2-6}
    &	425056	&	0,4	&	0,01857	&	0,939773	&	0,060227	\\ \cline{2-6}
    &	283370,67	&	0,6	&	0,041783	&	0,864488	&	0,135512	\\ \cline{2-6}
    &	212528	&	0,8	&	0,111421	&	0,759091	&	0,240909	\\
    \newpage
    \multirow{4}{*}{M1-M6}
    &	1262144	&	0,2	&	0,00469	&	0,989859	&	0,010141	\\ \cline{2-6}
    &	631072	&	0,4	&	0,012508	&	0,959434	&	0,040566	\\ \cline{2-6}
    &	420714,67	&	0,6	&	0,028143	&	0,908727	&	0,091273	\\ \cline{2-6}
    &	315536	&	0,8	&	0,075047	&	0,837736	&	0,162264	\\ \hline
    \multirow{4}{*}{M1-M7}
    &	1018240	&	0,2	&	0,005814	&	0,987429	&	0,012571	\\ \cline{2-6}
    &	509120	&	0,4	&	0,015504	&	0,949717	&	0,050283	\\ \cline{2-6}
    &	339413,33	&	0,6	&	0,034884	&	0,886864	&	0,113136	\\ \cline{2-6}
    &	254560	&	0,8	&	0,093023	&	0,798869	&	0,201131	\\ \hline
    \multirow{4}{*}{M1-M8}
    &	897472	&	0,2	&	0,006596	&	0,985738	&	0,014262	\\ \cline{2-6}
    &	448736	&	0,4	&	0,01759	&	0,942951	&	0,057049	\\ \cline{2-6}
    &	299157,33	&	0,6	&	0,039578	&	0,871639	&	0,128361	\\ \cline{2-6}
    &	224368	&	0,8	&	0,105541	&	0,771803	&	0,228197	\\ \hline
    \multirow{4}{*}{M1-M9}
    &	873792	&	0,2	&	0,006775	&	0,985351	&	0,014649	\\ \cline{2-6}
    &	436896	&	0,4	&	0,018067	&	0,941405	&	0,058595	\\ \cline{2-6}
    &	291264	&	0,6	&	0,04065	&	0,868161	&	0,131839	\\ \cline{2-6}
    &	218448	&	0,8	&	0,108401	&	0,765619	&	0,234381	\\ \hline
    \multirow{4}{*}{M1-M10}
    &	776704	&	0,2	&	0,007622	&	0,98352	&	0,01648	\\ \cline{2-6}
    &	388352	&	0,4	&	0,020325	&	0,93408	&	0,06592	\\ \cline{2-6}
    &	258901,33	&	0,6	&	0,045732	&	0,851681	&	0,148319	\\ \cline{2-6}
    &	194176	&	0,8	&	0,121951	&	0,736322	&	0,263678	\\ \hline
    \multirow{4}{*}{M2-M1}
     &	291264	&	0,2	&	0,020325	&	0,956054	&	0,043946 \\ \cline{2-6}
     &	145632	&	0,4	&	0,054201	&	0,824214	&	0,175786 \\ \cline{2-6}
     &	97088	&	0,6	&	0,121951	&	0,604483	&	0,395517 \\ \cline{2-6}
     &	72816	&	0,8	&	0,325203	&	0,296858	&	0,703142 \\ \hline
    \multirow{4}{*}{M2-M3}
    &	776704	&	0,2	&	0,007622	&	0,98352	&	0,01648	\\ \cline{2-6}
    &	388352	&	0,4	&	0,020325	&	0,93408	&	0,06592	\\ \cline{2-6}
    &	258901,33	&	0,6	&	0,045732	&	0,851681	&	0,148319	\\ \cline{2-6}
    &	194176	&	0,8	&	0,121951	&	0,736322	&	0,263678	\\ \hline
    \multirow{4}{*}{M2-M4}
    &	1018240	&	0,2	&	0,005814	&	0,987429	&	0,012571	\\ \cline{2-6}
    &	509120	&	0,4	&	0,015504	&	0,949717	&	0,050283	\\ \cline{2-6}
    &	339413,33	&	0,6	&	0,034884	&	0,886864	&	0,113136	\\ \cline{2-6}
    &	254560	&	0,8	&	0,093023	&	0,798869	&	0,201131	\\ \hline
    \multirow{4}{*}{M2-M5}
    &	850112	&	0,2	&	0,006964	&	0,984943	&	0,015057	\\ \cline{2-6}
    &	425056	&	0,4	&	0,01857	&	0,939773	&	0,060227	\\ \cline{2-6}
    &	283370,67	&	0,6	&	0,041783	&	0,864488	&	0,135512	\\ \cline{2-6}
    &	212528	&	0,8	&	0,111421	&	0,759091	&	0,240909	\\ \hline
    \multirow{4}{*}{M2-M6}
    &	1262144	&	0,2	&	0,00469	&	0,989859	&	0,010141	\\ \cline{2-6}
    &	631072	&	0,4	&	0,012508	&	0,959434	&	0,040566	\\ \cline{2-6}
    &	420714,67	&	0,6	&	0,028143	&	0,908727	&	0,091273	\\ \cline{2-6}
    &	315536	&	0,8	&	0,075047	&	0,837736	&	0,162264	\\ \hline
    \multirow{4}{*}{M2-M7}
    &	1018240	&	0,2	&	0,005814	&	0,987429	&	0,012571	\\ \cline{2-6}
    &	509120	&	0,4	&	0,015504	&	0,949717	&	0,050283	\\ \cline{2-6}
    &	339413,33	&	0,6	&	0,034884	&	0,886864	&	0,113136	\\ \cline{2-6}
    &	254560	&	0,8	&	0,093023	&	0,798869	&	0,201131	\\ 
    \newpage
    \multirow{4}{*}{M2-M8}
    &	897472	&	0,2	&	0,006596	&	0,985738	&	0,014262	\\ \cline{2-6}
    &	448736	&	0,4	&	0,01759	&	0,942951	&	0,057049	\\ \cline{2-6}
    &	299157,33	&	0,6	&	0,039578	&	0,871639	&	0,128361	\\ \cline{2-6}
    &	224368	&	0,8	&	0,105541	&	0,771803	&	0,228197	\\ \hline
    \multirow{4}{*}{M2-M9}
    &	873792	&	0,2	&	0,006775	&	0,985351	&	0,014649	\\ \cline{2-6}
    &	436896	&	0,4	&	0,018067	&	0,941405	&	0,058595	\\ \cline{2-6}
    &	291264	&	0,6	&	0,04065	&	0,868161	&	0,131839	\\ \cline{2-6}
    &	218448	&	0,8	&	0,108401	&	0,765619	&	0,234381	\\ \hline
    \multirow{4}{*}{M2-M10}
    &	776704	&	0,2	&	0,007622	&	0,98352	&	0,01648	\\ \cline{2-6}
    &	388352	&	0,4	&	0,020325	&	0,93408	&	0,06592	\\ \cline{2-6}
    &	258901,33	&	0,6	&	0,045732	&	0,851681	&	0,148319	\\ \cline{2-6}
    &	194176	&	0,8	&	0,121951	&	0,736322	&	0,263678	\\ \hline
    \multirow{4}{*}{M3-M1}
    &	776704	&	0,2	&	0,007622	&	0,98352	&	0,01648	\\ \cline{2-6}
    &	388352	&	0,4	&	0,020325	&	0,93408	&	0,06592	\\ \cline{2-6}
    &	258901,33	&	0,6	&	0,045732	&	0,851681	&	0,148319	\\ \cline{2-6}
    &	194176	&	0,8	&	0,121951	&	0,736322	&	0,263678	\\ \hline
    \multirow{4}{*}{M3-M2}
    &	776704	&	0,2	&	0,007622	&	0,98352	&	0,01648	\\ \cline{2-6}
    &	388352	&	0,4	&	0,020325	&	0,93408	&	0,06592	\\ \cline{2-6}
    &	258901,33	&	0,6	&	0,045732	&	0,851681	&	0,148319	\\ \cline{2-6}
    &	194176	&	0,8	&	0,121951	&	0,736322	&	0,263678	\\ \hline
    \multirow{4}{*}{M3-M4}
    &	2718464	&	0,2	&	0,002178	&	0,995291	&	0,004709	\\ \cline{2-6}
    &	1359232	&	0,4	&	0,005807	&	0,981166	&	0,018834	\\ \cline{2-6}
    &	906154,67	&	0,6	&	0,013066	&	0,957623	&	0,042377	\\ \cline{2-6}
    &	679616	&	0,8	&	0,034843	&	0,924663	&	0,075337	\\ \hline
    \multirow{4}{*}{M3-M5}
    &	2263808	&	0,2	&	0,002615	&	0,994346	&	0,005654	\\ \cline{2-6}
    &	1131904	&	0,4	&	0,006974	&	0,977383	&	0,022617	\\ \cline{2-6}
    &	754602,67	&	0,6	&	0,01569	&	0,949112	&	0,050888	\\ \cline{2-6}
    &	565952	&	0,8	&	0,041841	&	0,909533	&	0,090467	\\ \hline
    \multirow{4}{*}{M3-M6}
    &	3364928	&	0,2	&	0,001759	&	0,996196	&	0,003804	\\ \cline{2-6}
    &	1682464	&	0,4	&	0,004692	&	0,984784	&	0,015216	\\ \cline{2-6}
    &	1121642,67	&	0,6	&	0,010556	&	0,965764	&	0,034236	\\ \cline{2-6}
    &	841232	&	0,8	&	0,028149	&	0,939137	&	0,060863	\\ \hline
    \multirow{4}{*}{M3-M7}
    &	2718464	&	0,2	&	0,002178	&	0,995291	&	0,004709	\\ \cline{2-6}
    &	1359232	&	0,4	&	0,005807	&	0,981166	&	0,018834	\\ \cline{2-6}
    &	906154,67	&	0,6	&	0,013066	&	0,957623	&	0,042377	\\ \cline{2-6}
    &	679616	&	0,8	&	0,034843	&	0,924663	&	0,075337	\\ \hline
    \multirow{4}{*}{M3-M8}
    &	2394048	&	0,2	&	0,002473	&	0,994653	&	0,005347	\\ \cline{2-6}
    &	1197024	&	0,4	&	0,006594	&	0,978614	&	0,021386	\\ \cline{2-6}
    &	798016	&	0,6	&	0,014837	&	0,951881	&	0,048119	\\ \cline{2-6}
    &	598512	&	0,8	&	0,039565	&	0,914455	&	0,085545	\\ \hline
    \multirow{4}{*}{M3-M9}
    &	2330112	&	0,2	&	0,002541	&	0,994507	&	0,005493	\\ \cline{2-6}
    &	1165056	&	0,4	&	0,006775	&	0,978027	&	0,021973	\\ \cline{2-6}
    &	776704	&	0,6	&	0,015244	&	0,95056	&	0,04944	\\ \cline{2-6}
    &	582528	&	0,8	&	0,04065	&	0,912107	&	0,087893	\\
    \newpage
    \multirow{4}{*}{M3-M10}
    &	2069632	&	0,2	&	0,00286	&	0,993815	&	0,006185	\\ \cline{2-6}
    &	1034816	&	0,4	&	0,007628	&	0,975261	&	0,024739	\\ \cline{2-6}
    &	689877,33	&	0,6	&	0,017162	&	0,944338	&	0,055662	\\ \cline{2-6}
    &	517408	&	0,8	&	0,045767	&	0,901045	&	0,098955	\\ \hline
    \multirow{4}{*}{M4-M1}
    &	1018240	&	0,2	&	0,005814	&	0,987429	&	0,012571	\\ \cline{2-6}
    &	509120	&	0,4	&	0,015504	&	0,949717	&	0,050283	\\ \cline{2-6}
    &	339413,33	&	0,6	&	0,034884	&	0,886864	&	0,113136	\\ \cline{2-6}
    &	254560	&	0,8	&	0,093023	&	0,798869	&	0,201131	\\ \hline
    \multirow{4}{*}{M4-M2}
    &	1018240	&	0,2	&	0,005814	&	0,987429	&	0,012571	\\ \cline{2-6}
    &	509120	&	0,4	&	0,015504	&	0,949717	&	0,050283	\\ \cline{2-6}
    &	339413,33	&	0,6	&	0,034884	&	0,886864	&	0,113136	\\ \cline{2-6}
    &	254560	&	0,8	&	0,093023	&	0,798869	&	0,201131	\\ \hline
    \multirow{4}{*}{M4-M3}
    &	2718464	&	0,2	&	0,002178	&	0,995291	&	0,004709	\\ \cline{2-6}
    &	1359232	&	0,4	&	0,005807	&	0,981166	&	0,018834	\\ \cline{2-6}
    &	906154,67	&	0,6	&	0,013066	&	0,957623	&	0,042377	\\ \cline{2-6}
    &	679616	&	0,8	&	0,034843	&	0,924663	&	0,075337	\\ \hline
    \multirow{4}{*}{M4-M5}
    &	2971840	&	0,2	&	0,001992	&	0,995693	&	0,004307	\\ \cline{2-6}
    &	1485920	&	0,4	&	0,005312	&	0,982772	&	0,017228	\\ \cline{2-6}
    &	990613,33	&	0,6	&	0,011952	&	0,961236	&	0,038764	\\ \cline{2-6}
    &	742960	&	0,8	&	0,031873	&	0,931086	&	0,068914	\\ \hline
    \multirow{4}{*}{M4-M6}
    &	4416320	&	0,2	&	0,00134	&	0,997102	&	0,002898	\\ \cline{2-6}
    &	2208160	&	0,4	&	0,003575	&	0,988407	&	0,011593	\\ \cline{2-6}
    &	1472106,67	&	0,6	&	0,008043	&	0,973915	&	0,026085	\\ \cline{2-6}
    &	1104080	&	0,8	&	0,021448	&	0,953627	&	0,046373	\\ \hline
    \multirow{4}{*}{M4-M7}
    &	3566208	&	0,2	&	0,00166	&	0,996411	&	0,003589	\\ \cline{2-6}
    &	1783104	&	0,4	&	0,004427	&	0,985643	&	0,014357	\\ \cline{2-6}
    &	1188736	&	0,6	&	0,00996	&	0,967697	&	0,032303	\\ \cline{2-6}
    &	891552	&	0,8	&	0,02656	&	0,942572	&	0,057428	\\ \hline
    \multirow{4}{*}{M4-M8}
    &	3142336	&	0,2	&	0,001884	&	0,995927	&	0,004073	\\ \cline{2-6}
    &	1571168	&	0,4	&	0,005024	&	0,983706	&	0,016294	\\ \cline{2-6}
    &	1047445,33	&	0,6	&	0,011304	&	0,963339	&	0,036661	\\ \cline{2-6}
    &	785584	&	0,8	&	0,030143	&	0,934826	&	0,065174	\\ \hline
    \multirow{4}{*}{M4-M9}
    &	3057088	&	0,2	&	0,001936	&	0,995813	&	0,004187	\\ \cline{2-6}
    &	1528544	&	0,4	&	0,005164	&	0,983252	&	0,016748	\\ \cline{2-6}
    &	1019029,33	&	0,6	&	0,011619	&	0,962317	&	0,037683	\\ \cline{2-6}
    &	764272	&	0,8	&	0,030984	&	0,933008	&	0,066992	\\ \hline
    \multirow{4}{*}{M4-M10}
    &	2718464	&	0,2	&	0,002178	&	0,995291	&	0,004709	\\ \cline{2-6}
    &	1359232	&	0,4	&	0,005807	&	0,981166	&	0,018834	\\ \cline{2-6}
    &	906154,67	&	0,6	&	0,013066	&	0,957623	&	0,042377	\\ \cline{2-6}
    &	679616	&	0,8	&	0,034843	&	0,924663	&	0,075337	\\ \hline
    \multirow{4}{*}{M5-M1}
    &	850112	&	0,2	&	0,006964	&	0,984943	&	0,015057	\\ \cline{2-6}
    &	425056	&	0,4	&	0,01857	&	0,939773	&	0,060227	\\ \cline{2-6}
    &	283370,67	&	0,6	&	0,041783	&	0,864488	&	0,135512	\\ \cline{2-6}
    &	212528	&	0,8	&	0,111421	&	0,759091	&	0,240909	\\
    \newpage
    \multirow{4}{*}{M5-M2}
    &	850112	&	0,2	&	0,006964	&	0,984943	&	0,015057	\\ \cline{2-6}
    &	425056	&	0,4	&	0,01857	&	0,939773	&	0,060227	\\ \cline{2-6}
    &	283370,67	&	0,6	&	0,041783	&	0,864488	&	0,135512	\\ \cline{2-6}
    &	212528	&	0,8	&	0,111421	&	0,759091	&	0,240909	\\ \hline
    \multirow{4}{*}{M5-M3}
    &	2263808	&	0,2	&	0,002615	&	0,994346	&	0,005654	\\ \cline{2-6}
    &	1131904	&	0,4	&	0,006974	&	0,977383	&	0,022617	\\ \cline{2-6}
    &	754602,67	&	0,6	&	0,01569	&	0,949112	&	0,050888	\\ \cline{2-6}
    &	565952	&	0,8	&	0,041841	&	0,909533	&	0,090467	\\ \hline
    \multirow{4}{*}{M5-M4}
    &	2971840	&	0,2	&	0,001992	&	0,995693	&	0,004307	\\ \cline{2-6}
    &	1485920	&	0,4	&	0,005312	&	0,982772	&	0,017228	\\ \cline{2-6}
    &	990613,33	&	0,6	&	0,011952	&	0,961236	&	0,038764	\\ \cline{2-6}
    &	742960	&	0,8	&	0,031873	&	0,931086	&	0,068914	\\ \hline
    \multirow{4}{*}{M5-M6}
    &	3679872	&	0,2	&	0,001609	&	0,996522	&	0,003478	\\ \cline{2-6}
    &	1839936	&	0,4	&	0,00429	&	0,986086	&	0,013914	\\ \cline{2-6}
    &	1226624	&	0,6	&	0,009653	&	0,968695	&	0,031305	\\ \cline{2-6}
    &	919968	&	0,8	&	0,02574	&	0,944346	&	0,055654	\\ \hline
    \multirow{4}{*}{M5-M7}
    &	2971840	&	0,2	&	0,001992	&	0,995693	&	0,004307	\\ \cline{2-6}
    &	1485920	&	0,4	&	0,005312	&	0,982772	&	0,017228	\\ \cline{2-6}
    &	990613,33	&	0,6	&	0,011952	&	0,961236	&	0,038764	\\ \cline{2-6}
    &	742960	&	0,8	&	0,031873	&	0,931086	&	0,068914	\\ \hline
    \multirow{4}{*}{M5-M8}
    &	2619008	&	0,2	&	0,00226	&	0,995113	&	0,004887	\\ \cline{2-6}
    &	1309504	&	0,4	&	0,006028	&	0,980451	&	0,019549	\\ \cline{2-6}
    &	873002,67	&	0,6	&	0,013562	&	0,956014	&	0,043986	\\ \cline{2-6}
    &	654752	&	0,8	&	0,036166	&	0,921802	&	0,078198	\\ \hline
    \multirow{4}{*}{M5-M9}
    &	2547968	&	0,2	&	0,002323	&	0,994976	&	0,005024	\\ \cline{2-6}
    &	1273984	&	0,4	&	0,006196	&	0,979906	&	0,020094	\\ \cline{2-6}
    &	849322,67	&	0,6	&	0,013941	&	0,954788	&	0,045212	\\ \cline{2-6}
    &	636992	&	0,8	&	0,037175	&	0,919622	&	0,080378	\\ \hline
    \multirow{4}{*}{M5-M10}
    &	2263808	&	0,2	&	0,002615	&	0,994346	&	0,005654	\\ \cline{2-6}
    &	1131904	&	0,4	&	0,006974	&	0,977383	&	0,022617	\\ \cline{2-6}
    &	754602,67	&	0,6	&	0,01569	&	0,949112	&	0,050888	\\ \cline{2-6}
    &	565952	&	0,8	&	0,041841	&	0,909533	&	0,090467	\\ \hline
    \multirow{4}{*}{M6-M1}
    &	1262144	&	0,2	&	0,00469	&	0,989859	&	0,010141	\\ \cline{2-6}
    &	631072	&	0,4	&	0,012508	&	0,959434	&	0,040566	\\ \cline{2-6}
    &	420714,67	&	0,6	&	0,028143	&	0,908727	&	0,091273	\\ \cline{2-6}
    &	315536	&	0,8	&	0,075047	&	0,837736	&	0,162264	\\ \hline
    \multirow{4}{*}{M6-M2}
    &	1262144	&	0,2	&	0,00469	&	0,989859	&	0,010141	\\ \cline{2-6}
    &	631072	&	0,4	&	0,012508	&	0,959434	&	0,040566	\\ \cline{2-6}
    &	420714,67	&	0,6	&	0,028143	&	0,908727	&	0,091273	\\ \cline{2-6}
    &	315536	&	0,8	&	0,075047	&	0,837736	&	0,162264	\\ \hline
    \multirow{4}{*}{M6-M3}
    &	3364928	&	0,2	&	0,001759	&	0,996196	&	0,003804	\\ \cline{2-6}
    &	1682464	&	0,4	&	0,004692	&	0,984784	&	0,015216	\\ \cline{2-6}
    &	1121642,67	&	0,6	&	0,010556	&	0,965764	&	0,034236	\\ \cline{2-6}
    &	841232	&	0,8	&	0,028149	&	0,939137	&	0,060863	\\
    \newpage
    \multirow{4}{*}{M6-M4}
    &	4416320	&	0,2	&	0,00134	&	0,997102	&	0,002898	\\ \cline{2-6}
    &	2208160	&	0,4	&	0,003575	&	0,988407	&	0,011593	\\ \cline{2-6}
    &	1472106,67	&	0,6	&	0,008043	&	0,973915	&	0,026085	\\ \cline{2-6}
    &	1104080	&	0,8	&	0,021448	&	0,953627	&	0,046373	\\ \hline
    \multirow{4}{*}{M6-M5}
    &	3679872	&	0,2	&	0,001609	&	0,996522	&	0,003478	\\ \cline{2-6}
    &	1839936	&	0,4	&	0,00429	&	0,986086	&	0,013914	\\ \cline{2-6}
    &	1226624	&	0,6	&	0,009653	&	0,968695	&	0,031305	\\ \cline{2-6}
    &	919968	&	0,8	&	0,02574	&	0,944346	&	0,055654	\\ \hline
    \multirow{4}{*}{M6-M7}
    &	4416320	&	0,2	&	0,00134	&	0,997102	&	0,002898	\\ \cline{2-6}
    &	2208160	&	0,4	&	0,003575	&	0,988407	&	0,011593	\\ \cline{2-6}
    &	1472106,67	&	0,6	&	0,008043	&	0,973915	&	0,026085	\\ \cline{2-6}
    &	1104080	&	0,8	&	0,021448	&	0,953627	&	0,046373	\\ \hline
    \multirow{4}{*}{M6-M8}
    &	3890624	&	0,2	&	0,001522	&	0,99671	&	0,00329	\\ \cline{2-6}
    &	1945312	&	0,4	&	0,004058	&	0,98684	&	0,01316	\\ \cline{2-6}
    &	1296874,67	&	0,6	&	0,00913	&	0,97039	&	0,02961	\\ \cline{2-6}
    &	972656	&	0,8	&	0,024346	&	0,947361	&	0,052639	\\ \hline
    \multirow{4}{*}{M6-M9}
    &	3784064	&	0,2	&	0,001564	&	0,996617	&	0,003383	\\ \cline{2-6}
    &	1892032	&	0,4	&	0,004172	&	0,98647	&	0,01353	\\ \cline{2-6}
    &	1261354,67	&	0,6	&	0,009387	&	0,969557	&	0,030443	\\ \cline{2-6}
    &	946016	&	0,8	&	0,025031	&	0,945878	&	0,054122	\\ \hline
    \multirow{4}{*}{M6-M10}
    &	3364928	&	0,2	&	0,001759	&	0,996196	&	0,003804	\\ \cline{2-6}
    &	1682464	&	0,4	&	0,004692	&	0,984784	&	0,015216	\\ \cline{2-6}
    &	1121642,67	&	0,6	&	0,010556	&	0,965764	&	0,034236	\\ \cline{2-6}
    &	841232	&	0,8	&	0,028149	&	0,939137	&	0,060863	\\ \hline
    \multirow{4}{*}{M7-M1}
    &	1018240	&	0,2	&	0,005814	&	0,987429	&	0,012571	\\ \cline{2-6}
    &	509120	&	0,4	&	0,015504	&	0,949717	&	0,050283	\\ \cline{2-6}
    &	339413,33	&	0,6	&	0,034884	&	0,886864	&	0,113136	\\ \cline{2-6}
    &	254560	&	0,8	&	0,093023	&	0,798869	&	0,201131	\\ \hline
    \multirow{4}{*}{M7-M2}
    &	1018240	&	0,2	&	0,005814	&	0,987429	&	0,012571	\\ \cline{2-6}
    &	509120	&	0,4	&	0,015504	&	0,949717	&	0,050283	\\ \cline{2-6}
    &	339413,33	&	0,6	&	0,034884	&	0,886864	&	0,113136	\\ \cline{2-6}
    &	254560	&	0,8	&	0,093023	&	0,798869	&	0,201131	\\ \hline
    \multirow{4}{*}{M7-M3}
    &	2718464	&	0,2	&	0,002178	&	0,995291	&	0,004709	\\ \cline{2-6}
    &	1359232	&	0,4	&	0,005807	&	0,981166	&	0,018834	\\ \cline{2-6}
    &	906154,67	&	0,6	&	0,013066	&	0,957623	&	0,042377	\\ \cline{2-6}
    &	679616	&	0,8	&	0,034843	&	0,924663	&	0,075337	\\ \hline
    \multirow{4}{*}{M7-M4}
    &	3566208	&	0,2	&	0,00166	&	0,996411	&	0,003589	\\ \cline{2-6}
    &	1783104	&	0,4	&	0,004427	&	0,985643	&	0,014357	\\ \cline{2-6}
    &	1188736	&	0,6	&	0,00996	&	0,967697	&	0,032303	\\ \cline{2-6}
    &	891552	&	0,8	&	0,02656	&	0,942572	&	0,057428	\\ \hline
    \multirow{4}{*}{M7-M5}
    &	2971840	&	0,2	&	0,001992	&	0,995693	&	0,004307	\\ \cline{2-6}
    &	1485920	&	0,4	&	0,005312	&	0,982772	&	0,017228	\\ \cline{2-6}
    &	990613,33	&	0,6	&	0,011952	&	0,961236	&	0,038764	\\ \cline{2-6}
    &	742960	&	0,8	&	0,031873	&	0,931086	&	0,068914	\\
    \newpage
    \multirow{4}{*}{M7-M6}
    &	4416320	&	0,2	&	0,00134	&	0,997102	&	0,002898	\\ \cline{2-6}
    &	2208160	&	0,4	&	0,003575	&	0,988407	&	0,011593	\\ \cline{2-6}
    &	1472106,67	&	0,6	&	0,008043	&	0,973915	&	0,026085	\\ \cline{2-6}
    &	1104080	&	0,8	&	0,021448	&	0,953627	&	0,046373	\\ \hline
    \multirow{4}{*}{M7-M8}
    &	3142336	&	0,2	&	0,001884	&	0,995927	&	0,004073	\\ \cline{2-6}
    &	1571168	&	0,4	&	0,005024	&	0,983706	&	0,016294	\\ \cline{2-6}
    &	1047445,33	&	0,6	&	0,011304	&	0,963339	&	0,036661	\\ \cline{2-6}
    &	785584	&	0,8	&	0,030143	&	0,934826	&	0,065174	\\ \hline
    \multirow{4}{*}{M7-M9}
    &	3057088	&	0,2	&	0,001936	&	0,995813	&	0,004187	\\ \cline{2-6}
    &	1528544	&	0,4	&	0,005164	&	0,983252	&	0,016748	\\ \cline{2-6}
    &	1019029,33	&	0,6	&	0,011619	&	0,962317	&	0,037683	\\ \cline{2-6}
    &	764272	&	0,8	&	0,030984	&	0,933008	&	0,066992	\\ \hline
    \multirow{4}{*}{M7-M10}
    &	2718464	&	0,2	&	0,002178	&	0,995291	&	0,004709	\\ \cline{2-6}
    &	1359232	&	0,4	&	0,005807	&	0,981166	&	0,018834	\\ \cline{2-6}
    &	906154,67	&	0,6	&	0,013066	&	0,957623	&	0,042377	\\ \cline{2-6}
    &	679616	&	0,8	&	0,034843	&	0,924663	&	0,075337	\\ \hline
    \multirow{4}{*}{M8-M1}
    &	897472	&	0,2	&	0,006596	&	0,985738	&	0,014262	\\ \cline{2-6}
    &	448736	&	0,4	&	0,01759	&	0,942951	&	0,057049	\\ \cline{2-6}
    &	299157,33	&	0,6	&	0,039578	&	0,871639	&	0,128361	\\ \cline{2-6}
    &	224368	&	0,8	&	0,105541	&	0,771803	&	0,228197	\\ \hline
    \multirow{4}{*}{M8-M2}
    &	897472	&	0,2	&	0,006596	&	0,985738	&	0,014262	\\ \cline{2-6}
    &	448736	&	0,4	&	0,01759	&	0,942951	&	0,057049	\\ \cline{2-6}
    &	299157,33	&	0,6	&	0,039578	&	0,871639	&	0,128361	\\ \cline{2-6}
    &	224368	&	0,8	&	0,105541	&	0,771803	&	0,228197	\\ \hline
    \multirow{4}{*}{M8-M3}
    &	2394048	&	0,2	&	0,002473	&	0,994653	&	0,005347	\\ \cline{2-6}
    &	1197024	&	0,4	&	0,006594	&	0,978614	&	0,021386	\\ \cline{2-6}
    &	798016	&	0,6	&	0,014837	&	0,951881	&	0,048119	\\ \cline{2-6}
    &	598512	&	0,8	&	0,039565	&	0,914455	&	0,085545	\\ \hline
    \multirow{4}{*}{M8-M4}
    &	3142336	&	0,2	&	0,001884	&	0,995927	&	0,004073	\\ \cline{2-6}
    &	1571168	&	0,4	&	0,005024	&	0,983706	&	0,016294	\\ \cline{2-6}
    &	1047445,33	&	0,6	&	0,011304	&	0,963339	&	0,036661	\\ \cline{2-6}
    &	785584	&	0,8	&	0,030143	&	0,934826	&	0,065174	\\ \hline
    \multirow{4}{*}{M8-M5}
    &	2619008	&	0,2	&	0,00226	&	0,995113	&	0,004887	\\ \cline{2-6}
    &	1309504	&	0,4	&	0,006028	&	0,980451	&	0,019549	\\ \cline{2-6}
    &	873002,67	&	0,6	&	0,013562	&	0,956014	&	0,043986	\\ \cline{2-6}
    &	654752	&	0,8	&	0,036166	&	0,921802	&	0,078198	\\ \hline
    \multirow{4}{*}{M8-M6}
    &	3890624	&	0,2	&	0,001522	&	0,99671	&	0,00329	\\ \cline{2-6}
    &	1945312	&	0,4	&	0,004058	&	0,98684	&	0,01316	\\ \cline{2-6}
    &	1296874,67	&	0,6	&	0,00913	&	0,97039	&	0,02961	\\ \cline{2-6}
    &	972656	&	0,8	&	0,024346	&	0,947361	&	0,052639	\\ \hline
    \multirow{4}{*}{M8-M7}
    &	3142336	&	0,2	&	0,001884	&	0,995927	&	0,004073	\\ \cline{2-6}
    &	1571168	&	0,4	&	0,005024	&	0,983706	&	0,016294	\\ \cline{2-6}
    &	1047445,33	&	0,6	&	0,011304	&	0,963339	&	0,036661	\\ \cline{2-6}
    &	785584	&	0,8	&	0,030143	&	0,934826	&	0,065174	\\
    \newpage
    \multirow{4}{*}{M8-M9}
    &	2692416	&	0,2	&	0,002199	&	0,995246	&	0,004754	\\ \cline{2-6}
    &	1346208	&	0,4	&	0,005863	&	0,980984	&	0,019016	\\ \cline{2-6}
    &	897472	&	0,6	&	0,013193	&	0,957213	&	0,042787	\\ \cline{2-6}
    &	673104	&	0,8	&	0,03518	&	0,923934	&	0,076066	\\ \hline
    \multirow{4}{*}{M8-M10}
    &	2394048	&	0,2	&	0,002473	&	0,994653	&	0,005347	\\ \cline{2-6}
    &	1197024	&	0,4	&	0,006594	&	0,978614	&	0,021386	\\ \cline{2-6}
    &	798016	&	0,6	&	0,014837	&	0,951881	&	0,048119	\\ \cline{2-6}
    &	598512	&	0,8	&	0,039565	&	0,914455	&	0,085545	\\ \hline
    \multirow{4}{*}{M9-M1}
    &	873792	&	0,2	&	0,006775	&	0,985351	&	0,014649	\\ \cline{2-6}
    &	436896	&	0,4	&	0,018067	&	0,941405	&	0,058595	\\ \cline{2-6}
    &	291264	&	0,6	&	0,04065	&	0,868161	&	0,131839	\\ \cline{2-6}
    &	218448	&	0,8	&	0,108401	&	0,765619	&	0,234381	\\ \hline
    \multirow{4}{*}{M9-M2}
    &	873792	&	0,2	&	0,006775	&	0,985351	&	0,014649	\\ \cline{2-6}
    &	436896	&	0,4	&	0,018067	&	0,941405	&	0,058595	\\ \cline{2-6}
    &	291264	&	0,6	&	0,04065	&	0,868161	&	0,131839	\\ \cline{2-6}
    &	218448	&	0,8	&	0,108401	&	0,765619	&	0,234381	\\ \hline
    \multirow{4}{*}{M9-M3}
    &	2330112	&	0,2	&	0,002541	&	0,994507	&	0,005493	\\ \cline{2-6}
    &	1165056	&	0,4	&	0,006775	&	0,978027	&	0,021973	\\ \cline{2-6}
    &	776704	&	0,6	&	0,015244	&	0,95056	&	0,04944	\\ \cline{2-6}
    &	582528	&	0,8	&	0,04065	&	0,912107	&	0,087893	\\ \hline
    \multirow{4}{*}{M9-M4}
    &	3057088	&	0,2	&	0,001936	&	0,995813	&	0,004187	\\ \cline{2-6}
    &	1528544	&	0,4	&	0,005164	&	0,983252	&	0,016748	\\ \cline{2-6}
    &	1019029,33	&	0,6	&	0,011619	&	0,962317	&	0,037683	\\ \cline{2-6}
    &	764272	&	0,8	&	0,030984	&	0,933008	&	0,066992	\\ \hline
    \multirow{4}{*}{M9-M5}
    &	2547968	&	0,2	&	0,002323	&	0,994976	&	0,005024	\\ \cline{2-6}
    &	1273984	&	0,4	&	0,006196	&	0,979906	&	0,020094	\\ \cline{2-6}
    &	849322,67	&	0,6	&	0,013941	&	0,954788	&	0,045212	\\ \cline{2-6}
    &	636992	&	0,8	&	0,037175	&	0,919622	&	0,080378	\\ \hline
    \multirow{4}{*}{M9-M6}
    &	3784064	&	0,2	&	0,001564	&	0,996617	&	0,003383	\\ \cline{2-6}
    &	1892032	&	0,4	&	0,004172	&	0,98647	&	0,01353	\\ \cline{2-6}
    &	1261354,67	&	0,6	&	0,009387	&	0,969557	&	0,030443	\\ \cline{2-6}
    &	946016	&	0,8	&	0,025031	&	0,945878	&	0,054122	\\ \hline
    \multirow{4}{*}{M9-M7}
    &	3057088	&	0,2	&	0,001936	&	0,995813	&	0,004187	\\ \cline{2-6}
    &	1528544	&	0,4	&	0,005164	&	0,983252	&	0,016748	\\ \cline{2-6}
    &	1019029,33	&	0,6	&	0,011619	&	0,962317	&	0,037683	\\ \cline{2-6}
    &	764272	&	0,8	&	0,030984	&	0,933008	&	0,066992	\\ \hline
    \multirow{4}{*}{M9-M8}
    &	2692416	&	0,2	&	0,002199	&	0,995246	&	0,004754	\\ \cline{2-6}
    &	1346208	&	0,4	&	0,005863	&	0,980984	&	0,019016	\\ \cline{2-6}
    &	897472	&	0,6	&	0,013193	&	0,957213	&	0,042787	\\ \cline{2-6}
    &	673104	&	0,8	&	0,03518	&	0,923934	&	0,076066	\\ \hline
    \multirow{4}{*}{M9-M10}
    &	2330112	&	0,2	&	0,002541	&	0,994507	&	0,005493	\\ \cline{2-6}
    &	1165056	&	0,4	&	0,006775	&	0,978027	&	0,021973	\\ \cline{2-6}
    &	776704	&	0,6	&	0,015244	&	0,95056	&	0,04944	\\ \cline{2-6}
    &	582528	&	0,8	&	0,04065	&	0,912107	&	0,087893	\\
    \newpage
    \multirow{4}{*}{M10-M1}
    &	776704	&	0,2	&	0,007622	&	0,98352	&	0,01648	\\ \cline{2-6}
    &	388352	&	0,4	&	0,020325	&	0,93408	&	0,06592	\\ \cline{2-6}
    &	258901,33	&	0,6	&	0,045732	&	0,851681	&	0,148319	\\ \cline{2-6}
    &	194176	&	0,8	&	0,121951	&	0,736322	&	0,263678	\\ \hline
    \multirow{4}{*}{M10-M2}
    &	776704	&	0,2	&	0,007622	&	0,98352	&	0,01648	\\ \cline{2-6}
    &	388352	&	0,4	&	0,020325	&	0,93408	&	0,06592	\\ \cline{2-6}
    &	258901,33	&	0,6	&	0,045732	&	0,851681	&	0,148319	\\ \cline{2-6}
    &	194176	&	0,8	&	0,121951	&	0,736322	&	0,263678	\\ \hline
    \multirow{4}{*}{M10-M3}
    &	2069632	&	0,2	&	0,00286	&	0,993815	&	0,006185	\\ \cline{2-6}
    &	1034816	&	0,4	&	0,007628	&	0,975261	&	0,024739	\\ \cline{2-6}
    &	689877,33	&	0,6	&	0,017162	&	0,944338	&	0,055662	\\ \cline{2-6}
    &	517408	&	0,8	&	0,045767	&	0,901045	&	0,098955	\\ \hline
    \multirow{4}{*}{M10-M4}
    &	2718464	&	0,2	&	0,002178	&	0,995291	&	0,004709	\\ \cline{2-6}
    &	1359232	&	0,4	&	0,005807	&	0,981166	&	0,018834	\\ \cline{2-6}
    &	906154,67	&	0,6	&	0,013066	&	0,957623	&	0,042377	\\ \cline{2-6}
    &	679616	&	0,8	&	0,034843	&	0,924663	&	0,075337	\\ \hline
    \multirow{4}{*}{M10-M5}
    &	2263808	&	0,2	&	0,002615	&	0,994346	&	0,005654	\\ \cline{2-6}
    &	1131904	&	0,4	&	0,006974	&	0,977383	&	0,022617	\\ \cline{2-6}
    &	754602,67	&	0,6	&	0,01569	&	0,949112	&	0,050888	\\ \cline{2-6}
    &	565952	&	0,8	&	0,041841	&	0,909533	&	0,090467	\\ \hline
    \multirow{4}{*}{M10-M6}
    &	3364928	&	0,2	&	0,001759	&	0,996196	&	0,003804	\\ \cline{2-6}
    &	1682464	&	0,4	&	0,004692	&	0,984784	&	0,015216	\\ \cline{2-6}
    &	1121642,67	&	0,6	&	0,010556	&	0,965764	&	0,034236	\\ \cline{2-6}
    &	841232	&	0,8	&	0,028149	&	0,939137	&	0,060863	\\ \hline
    \multirow{4}{*}{M10-M7}
    &	2718464	&	0,2	&	0,002178	&	0,995291	&	0,004709	\\ \cline{2-6}
    &	1359232	&	0,4	&	0,005807	&	0,981166	&	0,018834	\\ \cline{2-6}
    &	906154,67	&	0,6	&	0,013066	&	0,957623	&	0,042377	\\ \cline{2-6}
    &	679616	&	0,8	&	0,034843	&	0,924663	&	0,075337	\\ \hline
    \multirow{4}{*}{M10-M8}
    &	2394048	&	0,2	&	0,002473	&	0,994653	&	0,005347	\\ \cline{2-6}
    &	1197024	&	0,4	&	0,006594	&	0,978614	&	0,021386	\\ \cline{2-6}
    &	798016	&	0,6	&	0,014837	&	0,951881	&	0,048119	\\ \cline{2-6}
    &	598512	&	0,8	&	0,039565	&	0,914455	&	0,085545	\\ \hline
    \multirow{4}{*}{M10-M9}
    &	2330112	&	0,2	&	0,002541	&	0,994507	&	0,005493	\\ \cline{2-6}
    &	1165056	&	0,4	&	0,006775	&	0,978027	&	0,021973	\\ \cline{2-6}
    &	776704	&	0,6	&	0,015244	&	0,95056	&	0,04944	\\ \cline{2-6}
    &	582528	&	0,8	&	0,04065	&	0,912107	& 0,087893 \\ \hline
    \multicolumn{6}{c}{}

    \label{table: VoIP Parameters}
\end{longtable}
\setlength\extrarowheight{2pt}
\normalsize
\vspace{-\baselineskip}

Разлічым сеткавыя параметры для сумарнага знешняга трафіка VoIP пры дапамозе
праграмы <<Разлік прапускной здольнасці каналаў сувязі>>. Рэзультаты разліку
прадстаўленыя ў табліцы \ref{table: Sum traffic}.

\newpage

\begin{table}[!h]
    \caption{Сеткавыя параметры для сумарнага значэння трафіка VoIP}
    \begin{tabularx}{\textwidth}{|c|c|c|c|>{\centering\arraybackslash}X|}
        \hline
            Хуткасць, біт/с & Загрузка & Затрымка, с
            & \makecell[c]{Імавернасць своечасовай\\ дастаўкі}
            & \makecell[c]{Імавернасць\\ страт} \\
        \hline
            222873792	&	0,2		& 0,000027	&	0,999943	& 0,000057		\\
        \hline
            111436896	&	0,4		& 0,000071	&	0,99977	&	0,00023	\\
        \hline
            74291264	&	0,6		& 0,000159	&	0,999483	&	0,000517 \\
        \hline
            55718448	&	0,8		& 0,000425	&	0,999081	&	0,000919 \\
        \hline
    \end{tabularx}
    \label{table: Sum traffic}
\end{table}

Пабудуем графік залежнасці прапускной здольнасці ад імавернасці страт
для розных значэнняў загрузкі ($\rho$ = 0,2; 0,4; 0,6; 0,8).

\begin{figure}[h!]
    \centering
    \includegraphics[width=0.8\textwidth]{chartVoIP.png}
    \vspace{-1cm}
    \caption{Залежнасць прапускной здольнасці ад імавернасці страт
             для сумарнага знешняга трафіка VoIP}
    \label{chart:Sum traffic VoIP}
\end{figure}

У табліцы \ref{table:Summary VoIP table} прыведзены сеткавыя параметры
для VoIP пры $\rho$ = 0,8.

\begin{table}[!htp]
    \caption{Зводная табліца сеткавых параметраў для VoIP пры $\rho$ = 0,8}
    \begin{tabularx}{\textwidth}{|c|c|c|c|>{\centering\arraybackslash}X|}
        \hline
            Маршрутызатар
            & \makecell[c]{Сумарная\\ інтэнсіўнасць\\ нагрузкі,\\ пакетаў/с}
            & \makecell[c]{Сумарная\\ прапускная\\ здольнасць,\\ Мбіт/с}
            & Час затрымкі, c & \makecell[c]{Імавернасць страт} \\
        \hline
            M1 & 340,2 & 1,941168 &  0,325203 & 0,703142 \\
        \hline
            M2 & 340,2 & 1,941168 &  0,325203 & 0,703142 \\
        \hline
            M3 & 907,2 & 4,853216 & 0,121951 & 0,263678 \\
        \hline
            M4 & 1190,6 & 6,1568 & 0,093023 & 0,201131 \\
        \hline
            M5 & 992,2 & 5,254592 & 0,111421 & 0,240909 \\
        \hline
            M6 & 1474,2 & 7,360336 & 0,075047 & 0,162264 \\
        \hline
            M7 & 1190,6 & 6,1568 & 0,093023 & 0,201131 \\
        \hline
            M8 & 1048,9 & 5,51744 & 0,105541 & 0,228197 \\
        \hline
            M9 & 1020,6 & 5,386608 & 0,108401 & 0,234381 \\
        \hline
            M10 & 907,2 & 4,853216 & 0,121951 & 0,263678 \\
        \hline
    \end{tabularx}
    \label{table:Summary VoIP table}
\end{table}

\newpage

Разлічым сеткавыя параметры для карыстальнікаў IPTV.
У сетцы выкарыстоўваецца рэжым шматадраснай рассылкі.
Так як выкарыстоўваецца MPEG-4 з хуткасцю 24 Mбіт/с, то прапускная
здольнасць канала сувязі ад сервера да маршрутызатара пры
трансляцыі аднаго IPTV-канала складае 24 Мбіт/с.

Інтэнсіўнасць абслугоўвання пакетаў разлічваецца па наступнай формуле:
\begin{equation}
    \mu = \frac{C_k \cdot N_{\text{IPTV}}}{L_n},
\end{equation}
\begin{Explanation}
    \item[дзе] $C$ -- хуткасць (прапускная здольнасць), біт/с;
    \item $L_n$ -- аб'ём пакета, біт;
    \item $N_\text{IPTV}$ -- колькасць IPTV-каналаў
\end{Explanation}

Згодна з варыянтам $N_\text{IPTV} = 152$.

Разлічым інтэнсіўнасць абслугоўвання пакетаў:
\begin{equation*}
    \mu = \frac{24 \cdot 10^6 \cdot 152}{592 \cdot 8} = 770270{,}27\ \text{c}^{-1}
\end{equation*}

Інтэнсіўнасць уваходнай нагрузкі разлічваецца па формуле:
\begin{equation}
    \lambda = \mu \cdot \rho
\end{equation}

Разлічым інтэнсіўнасць уваходнай нагрузкі для розных значэнняў
загрузкі $\rho$:
\begin{align*}
    \lambda_1 &= \mu \cdot \rho_1 = 770270{,}27 \cdot 0{,}2 = 154054{,}05\ \text{пакетаў}/\text{с}; \\
    \lambda_2 &= \mu \cdot \rho_2 = 770270{,}27 \cdot 0{,}4 = 308108{,}11\ \text{пакетаў}/\text{с}; \\
    \lambda_3 &= \mu \cdot \rho_3 = 770270{,}27 \cdot 0{,}6 = 462162{,}16\ \text{пакетаў}/\text{с}; \\
    \lambda_4 &= \mu \cdot \rho_4 = 770270{,}27 \cdot 0{,}8 = 616216{,}22\ \text{пакетаў}/\text{с}.
\end{align*}

Ведаючы інтэнсіўнасць уваходнай нагрузкі, разлічым сеткавыя параметры
пры перадачы IPTV трафіка пры дапамозе праграмы
<<Разлік прапускной здольнасці каналаў сувязі>>.
Рэзультаты вылічэння занясем ў табліцу \ref{table: IPTV Parameters}.

\newpage

\begin{table}[htp]
    \caption{Сеткавыя параметры для IPTV на ўсe кірункі сувязі}

    \begin{tabularx}{\textwidth}{|>{\centering\arraybackslash}X
                                 |>{\centering\arraybackslash}m{3cm}
                                 |>{\centering\arraybackslash}X
                                 |>{\centering\arraybackslash}X
                                 |>{\centering\arraybackslash}X
                                 |>{\centering\arraybackslash}X|}
        \hline
        $\lambda$, пакетаў/с & $C$, біт/с & $\rho$ & $t_{\text{зат}}$, c
        & $P_{\text{сд}}$ & $P_{\text{c}}$ \\
        \hline
        \multirow{4}{*}{154054,05}
        &	3647999904	&	0,2	&	0,000002	&	0,999996	&	0,000004	\\ \cline{2-6}
        &	1823999952	&	0,4	&	0,000004	&	0,999986	&	0,000014	\\ \cline{2-6}
        &	1215999968	&	0,6	&	0,000010	&	0,999968	&	0,000032	\\ \cline{2-6}
        &	911999976	&	0,8	&	0,000026	&	0,999944	&	0,000056	\\ \hline
        \multirow{4}{*}{308108,11}
        &	7296000044,8	&	0,2	&	0,000001	&	0,999998	&	0,000002	\\ \cline{2-6}
        &	3648000022,4	&	0,4	&	0,000002	&	0,999993	&	0,000007	\\ \cline{2-6}
        &	2432000014,93	&	0,6	&	0,000005	&	0,999984	&	0,000016	\\ \cline{2-6}
        &	1824000011,2	&	0,8	&	0,000013	&	0,999972	&	0,000028	\\ \hline
        \multirow{4}{*}{462162,16}
        &	10943999948,8	&	0,2	&	0,000001	&	0,999999	&	0,000001	\\ \cline{2-6}
        &	5471999974,4	&	0,4	&	0,000001	&	0,999995	&	0,000005	\\ \cline{2-6}
        &	3647999982,93	&	0,6	&	0,000003	&	0,999989	&	0,000011	\\ \cline{2-6}
        &	2735999987,2	&	0,8	&	0,000009	&	0,999981	&	0,000019	\\ \hline
        \multirow{4}{*}{616216,22}
        &	14592000089,6	&	0,2	&	0,000000	&	0,999999	&	0,000001	\\ \cline{2-6}
        &	7296000044,8	&	0,4	&	0,000001	&	0,999996	&	0,000004	\\ \cline{2-6}
        &	4864000029,87	&	0,6	&	0,000002	&	0,999992	&	0,000008	\\ \cline{2-6}
        &	3648000022,4	&	0,8	&	0,000006	&	0,999986	&	0,000014	\\ \hline
    \end{tabularx}
    \label{table: IPTV Parameters}
\end{table}

Такім чынам, сумарнае значэнне інтэнсіўнасці ўваходнай нагрузкі будзе роўная:
\begin{equation*}
    \lambda = 9411,9 + 616216{,}22 = 625628{,}1, \text{пакетаў/с}
\end{equation*}

Далей запішам рэзультаты разліку сеткавых параметраў для сумарнага знешняга
трафіка VoIP і IPTV у табліцу \ref{table: Summary VoIP and IPTV}.

\begin{table}[!h]
    \caption{Разлік сеткавых параметраў для сумарнага знешняга трафіка
             VOIP і IPTV}
    \begin{tabularx}{\textwidth}{|>{\centering\arraybackslash}m{3cm}
                                 |>{\centering\arraybackslash}X
                                 |>{\centering\arraybackslash}X
                                 |>{\centering\arraybackslash}X
                                 |>{\centering\arraybackslash}X|}
        \hline
        $C$, біт/с & $\rho$ & $t_{\text{зат}}$, c
        & $P_{\text{сд}}$ & $P_{\text{c}}$ \\
        \hline
        14814873408	&	0,2	&	0,000000	&	0,999999	&	0,000001	\\ \hline
        7407436704	&	0,4	&	0,000001	&	0,999997	&	0,000003	\\ \hline
        4938291136	&	0,6	&	0,000002	&	0,999992	&	0,000008	\\ \hline
        3703718352	&	0,8	&	0,000006	&	0,999986	&	0,000014	\\ \hline
     \end{tabularx}
    \label{table: Summary VoIP and IPTV}
\end{table}

Пабудуем графік залежнасці прапускной здольнасці ад
імавернасці страт пры $\rho$ = 0,8.

\newpage

\begin{figure}[h!]
    \centering
    \includegraphics[width=\textwidth]{chartVoIPandIPTV.png}
    \vspace{-1cm}
    \caption{Залежнасць прапускной здольнасці ад імавернасці страт
             для сумарнага знешняга трафіка VoIP і IPTV}
    \label{chart:Sum traffic VoIP and IPTV}
\end{figure}

Для выбару прадукцыйнасці маршрутызатараў і каналаў сувязі паміж маршрутызатарамі
вылічым сумарную інтэнсіўнасць уваходнай нагрузкі VoIP і IPTV на кожны
маршрутызатар і сумарную прапускную здольнасць, вынікі запішам ў табліцу
\ref{table: Router Summary}.

\begin{table}[h!]
    \caption{Зводная табліца сеткавых параметраў для VoIP і IPTV пры $\rho$ = 0,8}
    \begin{tabularx}{\textwidth}{|c|c|c|c|>{\centering\arraybackslash}X|}
        \hline
            Маршрутызатар
            & \makecell[c]{Сумарная\\ інтэнсіўнасць\\ нагрузкі,\\ пакетаў/с}
            & \makecell[c]{Сумарная\\ прапускная\\ здольнасць,\\ Мбіт/с}
            & Час затрымкі, c & \makecell[c]{Імавернасць страт} \\
        \hline
            M1 & 616556,42 & 3649,94 &  0,325203 & 0,703142 \\
        \hline
            M2 & 616556,42 & 3649,94 &  0,325203 & 0,703142 \\
        \hline
            M3 & 617123,42 & 3652,85 & 0,121951 & 0,263678 \\
        \hline
            M4 & 617406,82 & 3654,16 & 0,093023 & 0,201131 \\
        \hline
            M5 & 617208,42 & 3653,25 & 0,111421 & 0,240909 \\
        \hline
            M6 & 617690,42 & 3655,36 & 0,075047 & 0,162264 \\
        \hline
            M7 & 6170406,82 & 3654,16 & 0,093023 & 0,201131 \\
        \hline
            M8 & 617265,12 & 3653,52 & 0,105541 & 0,228197 \\
        \hline
            M9 & 617236,82 & 3653,39 & 0,108401 & 0,234381 \\
        \hline
            M10 & 617123,42 & 3652,85 & 0,121951 & 0,263678 \\
        \hline
    \end{tabularx}
    \label{table: Router Summary}
\end{table}

Даныя для выбару канала сувязі ў залежнасці ад прапускной здольнасці запішам
у табліцу \ref{table: channels of communication}.

Зрабіўшы аналіз атрыманых сеткавых параметраў для кожнага
маршрутызатара можна вызначыць, што праектуемая сетка не зможа
забяспечыць магчымасць забеспячэння паслуг VoIP з добрай
якасцю ва ўсіх кірунках пры любой загрузцы сеткі
(з табліцы \ref{table: VoIP Parameters} бачна, што на кірунках
M1-M2 і M2-M1 час затрымкі роўны 20 мс і імавернасць страты роўная
4\%, што перавышае дапушчальныя значэнні для забеспячэння
добрай якасці паслуг).

У той жа час сеткавыя параметры для IPTV на ўсе кірункі сувязі
адпавядаюць норме пры любой загрузцы сеткі:
\begin{enumerate}
    \item час затрымкі пакетаў паміж двума суседнімі маршрутызатарамі
          не больш за 150 мс;
    \item імавернасць страты не больш за 1\%.
\end{enumerate}

Так як сеткавыя параметры для VoIP перавышаюць дапушчальныя межы,
неабходна прыняць дадатковыя меры падчас далейшага планавання
сеткі для паляпшэння сеткавых параметраў.
Для паляпшэння сеткавых параметраў на ўсіх кірунках неабходна
прыняць наступныя меры:
\begin{enumerate}
    \item выкарыстоўваць маршрутызатары з высокай прадукцыйнасцю.
          Асаблівую ўвагу звярнуць на выбар маршрутызатараў
          M1 і M2 (патрабуецца максімальная прадукцыйнасць);
    \item выкарыстоўваць кодэк з меншай хуткасцю;
    \item выкарыстоўваць пратакол маршрутызацыі з гібкай настройкай
          палітык маршрутызацыі, для памяншэння нагрузкі на
          маршрутызатары М1 і М2.
\end{enumerate}

Улічваючы выдатныя сеткавыя параметры для IPTV можна паменшыць
колькасць IPTV каналаў для аптымізацыі кошту пры праектаванні сеткі.

Таксама неабходна разглядзець магчымасць выкарыстоўвання тэхналогіі
агрэгацыі (Link Aggregation Layer 3) для замены тыпу канала сувязі з
10GE на 4x1GE, так як прапускная здольнасць
10GE (10 Гбіт/с) выкарыстоўваецца
не аптымальна пры атрыманай сумарнай прапускной здольнасці
праектуемай сеткі (3,5 Гбіт/с).
