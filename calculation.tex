\section{Разлік сеткавых параметраў праектуемай сеткі}

\subsection{Матэматычная мадэль разліку сеткавых параметраў}

\subsection{Разлік матрыцы інфармацыйнага прыцягнення}

Для разліку матрыцы інфармацыйнага прыцягнення скарыстаемся
праграмай генерацыі матрыцы прыцягнення (для сетак пакетнай камутацыі).

У аснове праграмы закладзены наступны алгарытм:
\begin{enumerate}
    \item задаецца колькасць камутатараў;
    \item задаецца колькасць IP-абанентаў для зоны абслугоўвання
          кожнага маршрутызатара;
    \item задаецца ўдзельная абаненцкая нагрузка,
          абаненцкая нагрузка і іншыя параметры;
    \item памнажэннем колькасці IP-абанентаў (па ўсіх зонах)
          на ўдзельную абаненцкую нагрузку знаходзіцца
          сумарны знешні трафік (у Эрлангах);
    \item атрыманыя хвіліна-займанні пералічваем у пакетызаваны
          трафік (пакетаў/с).
          Для гэтага памнажаецца сумарны трафік у Эрлангах на
          хуткасць працы кодэка МПУ (мовапераўтваральнага ўстройства),
          паўзы пры гэтым выключаюцца,
          г. зн. іх не кадзіруем. Дзяленнем выніку на
          аб'ём пакета знаходзіцца інтэнсіўнасць
          сумарнага ўваходнага патоку (пакетаў/с);
    \item расшчэпліваецца сумарны ўваходны паток (пакетаў/с) па
          кірунках сувязі з улікам нераўнамернасці размеркавання
          абанентаў па зонах абслугоўвання маршрутызатараў.
          Расшчэпліванне патоку <<B>> выконваецца ў адпаведнасці
          з вядомым метадам <<Equal Distribution>>, згодна з якім
          элемент матрыцы <<B[i,j]>> вызначаюцца перамножваннем
          абаненцкіх ёмістасцей маршрутызатараў, якія разглядаем,
          на сумарны трафік <<B>> і дзяленнем рэзультату на
          сумарную абаненцкую ёмістасць сеткі ў квадраце:
          $TR(I,J) = EMK(I) \cdot EMK(J) \cdot B / ES / ES.$
    \item вынікам працы з'яўляецца матрыцы інфармацыйнага прыцягнення ---
          квадратная таб\-лі\-ца размернасці NxN (N -- зыходная колькасць
          маршрутызатараў).
\end{enumerate}

У табліцы \ref{table: The number of subscribers} прадстаўлена
колькасць абанентаў на кожны маршрутызатар.

\begin{table}[htp]
    \caption{Колькасць абанентаў на маршрутызатары}
    \begin{tabularx}{\textwidth}{ | >{\centering\arraybackslash}X
                                  | >{\centering\arraybackslash}X
                                  | >{\centering\arraybackslash}X | }
    \hline
        Нумар маршрутызатара & Колькасць абанентаў & Усяго абанентаў \\
    \hline
        1 & 1200
        &
        \multirow{10}{*}{33200} \\
    \cline{1-2}
        2 & 1200 & \\
    \cline{1-2}
        3 & 3200 & \\
    \cline{1-2}
        4 & 4200 & \\
    \cline{1-2}
        5 & 3500 & \\
    \cline{1-2}
        6 & 5200 & \\
    \cline{1-2}
        7 & 4200 & \\
    \cline{1-2}
        8 & 3700 & \\
    \cline{1-2}
        9 & 3600 & \\
    \cline{1-2}
        10 & 3200 & \\
    \hline
    \end{tabularx}
    \label{table: The number of subscribers}
\end{table}

Зыходныя і прамежкавыя даныя, якія будуць выкарыстоўвацца для
разліку параметраў сеткі, прыведзеныя ў табліцы
\ref{table: Original data}

\begin{table}[htp]
    \caption{Зыходныя і прамежкавыя даныя}
    \begin{tabularx}{\textwidth}{ | p{0.36\textwidth}
                                  | >{\centering\arraybackslash}X
                                  | >{\centering\arraybackslash}X | }
        \hline
            \multicolumn{1}{|c|}{Найменне паказальніка}
            &
            Адзінка вымярэння
            &
            Значэнне паказальніка \\
        \hline
            Колькасць маршрутызатараў & & 10 \\
        \hline
            Сярэдняя працягласць размовы & секунд & 100 \\
        \hline
            \makecell[l]{Інтэнсіўнасць выклікаў (у гадзіну\\ ад абанента)}
            & &
            2 \\
        \hline
            \makecell[l]{Удзельная выходная абаненцкая \\ нагрузка}
            & Эрлангаў & 0,0556 \\
        \hline
            \makecell[l]{Сумарная ўваходная абаненцкая \\ нагрузка}
            & Эрлангаў & 1826 \\
        \hline
            Аб'ём пакета & байт & 592 \\
        \hline
            Хуткасць працы МПУ & біт/с & 16000 \\
        \hline
            Сярэдняя працягласць фанемы & секунд & 1,34 \\
        \hline
            Сярэдняя працягласць паўзы & секунд & 1,67 \\
        \hline
            \makecell[l]{Сярэдняя колькасць актыўных \\ перыядаў у размове}
            & &
            16 \\
        \hline
            Сумарны знешні трафік & пакетаў/с & 9411,9 \\
        \hline
    \end{tabularx}
    \label{table: Original data}
\end{table}

Матрыца інфармацыйнага прыцягнення, якая была атрыманая
пры дапамозе праграмы генерацыі матрыцы прыцягнення,
прадстаўленая ў табліцы \ref{table: Matrix}.

\begin{table}[htp]
    \caption{Матрыца інфармацыйнага прыцягнення}
    \begin{tabularx}{\textwidth}{ |>{\centering\arraybackslash}X
                                  |c|c|c|c|c|c|c|c|c|c|}
        \hline
        \makecell{Нумар\\ маршрутызатара}
            & M1   & M2   & M3    & M4    & M5    & M6    & M7    & M8    & M9    & M10 \\
        \hline
        M1  & 12,3 & 12,3 & 32,8  & 43,0  & 35,9  & 53,3  & 43,0  & 37,9  & 36,9  & 32,8 \\
        \hline
        M2  & 12,3 & 12,3 & 32,8  & 43,0  & 35,9  & 53,3  & 43,0  & 37,9  & 36,9  & 32,8 \\
        \hline
        M3  & 32,8 & 32,8 & 87,4  & 114,8 & 95,6  & 142,1 & 114,8 & 101,1 & 98,4  & 87,4 \\
        \hline
        M4  & 43,0 & 43,0 & 114,8 & 150,6 & 125,5 & 186,5 & 150,6 & 132,7 & 129,1 & 114,8 \\
        \hline
        M5  & 35,9 & 35,9 & 95,6  & 125,5 & 104,4 & 155,4 & 125,5 & 110,6 & 107,6 & 95,6 \\
        \hline
        M6  & 53,3 & 53,3 & 142,1 & 186,5 & 156,4 & 230,9 & 186,5 & 164,3 & 159,8 & 142,1 \\
        \hline
        M7  & 43,0 & 43,0 & 114,8 & 150,6 & 125,5 & 186,5 & 150,6 & 132,7 & 129,1 & 114,8 \\
        \hline
        M8  & 37,9 & 37,9 & 101,1 & 132,7 & 110,6 & 164,3 & 132,7 & 116,9 & 113,7 & 101,1 \\
        \hline
        M9  & 36,9 & 36,9 & 98,4  & 129,1 & 107,6 & 159,8 & 129,1 & 113,7 & 110,7 & 98,4 \\
        \hline
        M10 & 32,8 & 32,8 & 87,4  & 114,8 & 95,6  & 142,1 & 114,8 & 101,1 & 98,4  & 87,4 \\
        \hline
    \end{tabularx}
    \label{table: Matrix}
\end{table}

Калі прасуміраваць ўсе значэнні з матрыцы інфармацыйнага прыцягнення,
атрымаем сумарны знешні трафік роўны 9411,9 пакетаў/с,
што адпавядае зыходным даным з табліцы
\ref{table: Original data} і з'яўляецца аптымальнай матрыцай
інфармацыйнага прыцягнення з прыведзеных у выніках праграмы.

\subsection{Разлік канальнага рэсурсу праектуемай сеткі}

Для разліку канальнага рэсурсу праектуемай сеткі скарыстаемся
праграмай <<Разлік прапускной здольнасці каналаў сувязі>>.

Значэнне сеткавых параметраў атрымліваем пасля ўводу ў праграму
інтэнсіўнасці уваходнай нагрузкі (з МІП) для кожнага маршрутызатара
ў кірунку іншых маршрутызатараў.

Разлічым сеткавыя параметры для карыстальнікаў VoIP. Пры гэтым пад
увагу прымаюцца зададзеныя нормы для добрай якасці дастаўкі:
\begin{enumerate}
    \item затрымка пакета не павінна перавышаць 10 мс;
    \item імавернасць страты пакета не павінна перавышаць 3\%.
\end{enumerate}

Атрыманыя рэзультаты аформім у выглядзе табліцы
\ref{table: Parameters}, дзе:
\begin{enumerate}
    \item $C$ -- хуткасць (прапускная здольнасць), біт/с;
    \item $\rho$ -- загрузка;
    \item $t_{\text{зат}}$ -- затрымка;
    \item $P_{\text{сд}}$ -- імавернасць своечасовай дастаўкі;
    \item $P_{\text{с}}$ -- імавернасць страты.
\end{enumerate}

%\begin{longtable}{@{\extracolsep{\fill}}|c|l|l|l|c|c|}
\setlength\LTleft{0pt}
\setlength\LTright{0pt}
\begin{longtable}{|>{\centering\arraybackslash}m{0.195\textwidth}
                  |>{\centering\arraybackslash}m{0.15\textwidth}
                  |>{\centering\arraybackslash}m{0.05\textwidth}
                  |>{\centering\arraybackslash}m{0.15\textwidth}
                  |>{\centering\arraybackslash}m{0.15\textwidth}
                  |>{\centering\arraybackslash}m{0.15\textwidth}|}
    \caption{Сеткавыя параметры для маршрутызатараў
             на ўсе кірункі сувязі} \\
    \hline
    Накірунак сувязі & $C$, біт/с & $\rho$ & $t_{\text{зат}}$
    & $P_{\text{сд}}$ & $P_{\text{c}}$ \\
    \hline
    1 & 2 & 3 & 4 & 5 & 6 \\
    \hline
    \endfirsthead

    \multicolumn{6}{l}{\hspace{-0.2cm}Працяг табліцы \thetable} \\
    \hline
    1 & 2 & 3 & 4 & 5 & 6 \\
    \endhead

    \hline
    \endlastfoot

    \multirow{4}{*}{M1-M2}
     &	291264	&	0,2	&	0,020325	&	0,956054	&	0,043946 \\ \cline{2-6}
     &	145632	&	0,4	&	0,054201	&	0,824214	&	0,175786 \\ \cline{2-6}
     &	97088	&	0,6	&	0,121951	&	0,604483	&	0,395517 \\ \cline{2-6}
     &	72816	&	0,8	&	0,325203	&	0,296858	&	0,703142 \\ \hline
    \multirow{4}{*}{M1-M3}
    &	776704	&	0,2	&	0,007622	&	0,98352	&	0,01648	\\ \cline{2-6}
    &	388352	&	0,4	&	0,020325	&	0,93408	&	0,06592	\\ \cline{2-6}
    &	258901,33	&	0,6	&	0,045732	&	0,851681	&	0,148319	\\ \cline{2-6}
    &	194176	&	0,8	&	0,121951	&	0,736322	&	0,263678	\\ \hline
    \multirow{4}{*}{M1-M4}
    &	1018240	&	0,2	&	0,005814	&	0,987429	&	0,012571	\\ \cline{2-6}
    &	509120	&	0,4	&	0,015504	&	0,949717	&	0,050283	\\ \cline{2-6}
    &	339413,33	&	0,6	&	0,034884	&	0,886864	&	0,113136	\\ \cline{2-6}
    &	254560	&	0,8	&	0,093023	&	0,798869	&	0,201131	\\ \hline
    \multirow{4}{*}{M1-M5}
    &	850112	&	0,2	&	0,006964	&	0,984943	&	0,015057	\\ \cline{2-6}
    &	425056	&	0,4	&	0,01857	&	0,939773	&	0,060227	\\ \cline{2-6}
    &	283370,67	&	0,6	&	0,041783	&	0,864488	&	0,135512	\\ \cline{2-6}
    &	212528	&	0,8	&	0,111421	&	0,759091	&	0,240909	\\ \hline
    \multirow{4}{*}{M1-M6}
    &	1262144	&	0,2	&	0,00469	&	0,989859	&	0,010141	\\ \cline{2-6}
    &	631072	&	0,4	&	0,012508	&	0,959434	&	0,040566	\\ \cline{2-6}
    &	420714,67	&	0,6	&	0,028143	&	0,908727	&	0,091273	\\ \cline{2-6}
    &	315536	&	0,8	&	0,075047	&	0,837736	&	0,162264	\\ \hline
    \multirow{4}{*}{M1-M7}
    &	1018240	&	0,2	&	0,005814	&	0,987429	&	0,012571	\\ \cline{2-6}
    &	509120	&	0,4	&	0,015504	&	0,949717	&	0,050283	\\ \cline{2-6}
    &	339413,33	&	0,6	&	0,034884	&	0,886864	&	0,113136	\\ \cline{2-6}
    &	254560	&	0,8	&	0,093023	&	0,798869	&	0,201131	\\ \hline
    \multirow{4}{*}{M1-M8}
    &	897472	&	0,2	&	0,006596	&	0,985738	&	0,014262	\\ \cline{2-6}
    &	448736	&	0,4	&	0,01759	&	0,942951	&	0,057049	\\ \cline{2-6}
    &	299157,33	&	0,6	&	0,039578	&	0,871639	&	0,128361	\\ \cline{2-6}
    &	224368	&	0,8	&	0,105541	&	0,771803	&	0,228197	\\ \hline
    \multirow{4}{*}{M1-M9}
    &	873792	&	0,2	&	0,006775	&	0,985351	&	0,014649	\\ \cline{2-6}
    &	436896	&	0,4	&	0,018067	&	0,941405	&	0,058595	\\ \cline{2-6}
    &	291264	&	0,6	&	0,04065	&	0,868161	&	0,131839	\\ \cline{2-6}
    &	218448	&	0,8	&	0,108401	&	0,765619	&	0,234381	\\ \hline
    \multirow{4}{*}{M1-M10}
    &	776704	&	0,2	&	0,007622	&	0,98352	&	0,01648	\\ \cline{2-6}
    &	388352	&	0,4	&	0,020325	&	0,93408	&	0,06592	\\ \cline{2-6}
    &	258901,33	&	0,6	&	0,045732	&	0,851681	&	0,148319	\\ \cline{2-6}
    &	194176	&	0,8	&	0,121951	&	0,736322	&	0,263678	\\ \hline
    \multirow{4}{*}{M2-M1}
     &	291264	&	0,2	&	0,020325	&	0,956054	&	0,043946 \\ \cline{2-6}
     &	145632	&	0,4	&	0,054201	&	0,824214	&	0,175786 \\ \cline{2-6}
     &	97088	&	0,6	&	0,121951	&	0,604483	&	0,395517 \\ \cline{2-6}
     &	72816	&	0,8	&	0,325203	&	0,296858	&	0,703142 \\ \hline
    \multirow{4}{*}{M2-M3}
    &	776704	&	0,2	&	0,007622	&	0,98352	&	0,01648	\\ \cline{2-6}
    &	388352	&	0,4	&	0,020325	&	0,93408	&	0,06592	\\ \cline{2-6}
    &	258901,33	&	0,6	&	0,045732	&	0,851681	&	0,148319	\\ \cline{2-6}
    &	194176	&	0,8	&	0,121951	&	0,736322	&	0,263678	\\ \hline
    \multirow{4}{*}{M2-M4}
    &	1018240	&	0,2	&	0,005814	&	0,987429	&	0,012571	\\ \cline{2-6}
    &	509120	&	0,4	&	0,015504	&	0,949717	&	0,050283	\\ \cline{2-6}
    &	339413,33	&	0,6	&	0,034884	&	0,886864	&	0,113136	\\ \cline{2-6}
    &	254560	&	0,8	&	0,093023	&	0,798869	&	0,201131	\\ \hline
    \multirow{4}{*}{M2-M5}
    &	850112	&	0,2	&	0,006964	&	0,984943	&	0,015057	\\ \cline{2-6}
    &	425056	&	0,4	&	0,01857	&	0,939773	&	0,060227	\\ \cline{2-6}
    &	283370,67	&	0,6	&	0,041783	&	0,864488	&	0,135512	\\ \cline{2-6}
    &	212528	&	0,8	&	0,111421	&	0,759091	&	0,240909	\\ \hline
    \multirow{4}{*}{M2-M6}
    &	1262144	&	0,2	&	0,00469	&	0,989859	&	0,010141	\\ \cline{2-6}
    &	631072	&	0,4	&	0,012508	&	0,959434	&	0,040566	\\ \cline{2-6}
    &	420714,67	&	0,6	&	0,028143	&	0,908727	&	0,091273	\\ \cline{2-6}
    &	315536	&	0,8	&	0,075047	&	0,837736	&	0,162264	\\ \hline
    \multirow{4}{*}{M2-M7}
    &	1018240	&	0,2	&	0,005814	&	0,987429	&	0,012571	\\ \cline{2-6}
    &	509120	&	0,4	&	0,015504	&	0,949717	&	0,050283	\\ \cline{2-6}
    &	339413,33	&	0,6	&	0,034884	&	0,886864	&	0,113136	\\ \cline{2-6}
    &	254560	&	0,8	&	0,093023	&	0,798869	&	0,201131	\\ \hline
    \multirow{4}{*}{M2-M8}
    &	897472	&	0,2	&	0,006596	&	0,985738	&	0,014262	\\ \cline{2-6}
    &	448736	&	0,4	&	0,01759	&	0,942951	&	0,057049	\\ \cline{2-6}
    &	299157,33	&	0,6	&	0,039578	&	0,871639	&	0,128361	\\ \cline{2-6}
    &	224368	&	0,8	&	0,105541	&	0,771803	&	0,228197	\\ \hline
    \multirow{4}{*}{M2-M9}
    &	873792	&	0,2	&	0,006775	&	0,985351	&	0,014649	\\ \cline{2-6}
    &	436896	&	0,4	&	0,018067	&	0,941405	&	0,058595	\\ \cline{2-6}
    &	291264	&	0,6	&	0,04065	&	0,868161	&	0,131839	\\ \cline{2-6}
    &	218448	&	0,8	&	0,108401	&	0,765619	&	0,234381	\\ \hline
    \multirow{4}{*}{M2-M10}
    &	776704	&	0,2	&	0,007622	&	0,98352	&	0,01648	\\ \cline{2-6}
    &	388352	&	0,4	&	0,020325	&	0,93408	&	0,06592	\\ \cline{2-6}
    &	258901,33	&	0,6	&	0,045732	&	0,851681	&	0,148319	\\ \cline{2-6}
    &	194176	&	0,8	&	0,121951	&	0,736322	&	0,263678	\\ \hline

    \label{table: Parameters}
\end{longtable}
