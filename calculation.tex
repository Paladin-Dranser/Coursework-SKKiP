\section{Разлік сеткавых параметраў праектуемай сеткі}

\subsection{Матэматычная мадэль разліку сеткавых параметраў}

\subsection{Разлік матрыцы інфармацыйнага прыцягнення}

Для разліку матрыцы інфармацыйнага прыцягнення скарыстаемся
праграмай генерацыі матрыцы прыцягнення (для сетак пакетнай камутацыі).

У аснове праграмы закладзены наступны алгарытм:
\begin{enumerate}
    \item задаецца колькасць камутатараў;
    \item задаецца колькасць IP-абанентаў для зоны абслугоўвання
          кожнага маршрутызатара;
    \item задаецца ўдзельная абаненцкая нагрука,
          абаненцкая нагрузка і іншыя параметры;
    \item памнажэннем колькасці IP-абанентаў (па ўсіх зонах)
          на ўдзельную абаненцкую нагрузку знаходзіцца
          сумарны знешні трафік (у Эрлангах);
    \item атрыманыя хвіліна-займанні пералічваем у пакетызаваны
          трафік (пакет/с).
          Для гэтага памнажаецца сумарны трафік у Эрлангах на
          хуткасць працы кодэка (паўзы пры гэтым выключаюцца,
          г. зн. іх не кадзіруем). Дзяленнем выніку на
          аб'ём пакета знаходзіцца інтэнсіўнасць
          сумарнага ўваходнага патоку (пакет/с);
    \item расшчэпліваецца сумарны ўваходны паток (пакет/с) па
          кірунках сувязі з улікам нераўнамернасці размеркавання
          абанентаў па зонах абслугоўвання маршрутызатараў;
    \item вынікам працы з'яўляецца матрыцы інфармацыйнага прыцягнення ---
          квадратная таб\-лі\-ца размернасці NxN (N - зыходная колькасць
          маршрутызатараў).
\end{enumerate}

У табліцы \ref{table: The number of subscribers} прадстаўлена
колькасць абанентаў на кожны маршрутызатар.

\begin{table}[htbp]
    \caption{Колькасць абанентаў на маршрутызатары}
    \begin{tabularx}{\textwidth}{ | >{\centering\arraybackslash}X
                                  | >{\centering\arraybackslash}X
                                  | >{\centering\arraybackslash}X | }
    \hline
        Нумар маршрутызатара & Колькасць абанентаў & Усяго абанентаў \\
    \hline
        1 & 1200
        &
        \multirow{10}{*}{33200} \\
    \cline{1-2}
        2 & 1200 & \\
    \cline{1-2}
        3 & 3200 & \\
    \cline{1-2}
        4 & 4200 & \\
    \cline{1-2}
        5 & 3500 & \\
    \cline{1-2}
        6 & 5200 & \\
    \cline{1-2}
        7 & 4200 & \\
    \cline{1-2}
        8 & 3700 & \\
    \cline{1-2}
        9 & 3600 & \\
    \cline{1-2}
        10 & 3200 & \\
    \hline
    \end{tabularx}
    \label{table: The number of subscribers}
\end{table}

\subsection{Разлік канальнага рэсурсу праектуемай сеткі}
