\sectionWithoutNumber{\prefacename}

На дадзены момант IP-сеткі з'яўляюцца найбольш распаўсюджанымі як
у Беларусі, так і іншых краінах свету. Асноўная прычына такой
папулярнасці IP-сетак --- разнастайнасці паслуг, якія рэалізуюцца
на IP-сетак. Такія сервісы як IP-тэлефанія, IP-тэлебачанне, відэа
па патрабаванню, VPN (Virtual Private Network) становяцца
неад'емнай часткай сучаснага свету.

VoIP (Voice over Internet Protocol) альбо IP-тэлефанія --- гэта
галасавая сувязь па пратаколу IP. Пад IP-тэлефаніей маецца на ўвазе
набор камунікацыйных пратаколаў, тэхналогій і метадаў, якія
забаспечваюць традыцыйныя для тэлефаніі паслугі, а таксама віэасувязь
праз сетку Інтэрнэт альбо іншым IP-сеткам.

На дадзены момант асноўным прызначэннем IP-тэлефаніі ёсць танныя альбо
бясплатныя міжгароднія і міжнародныя званкі. Аднак, асноўная перавага
VoIP для бізнесу --- гэта магчымасць пабудовы больш эфектыўнай сістэмы
карпаратыўных камунікацый з рознымі галасавымі сервісамі.
Эфектыўнасць такіх сістэм заключаецца ў наступным:
\begin{enumerate}
    \item больш простае і больш таннае ўкараненне;
    \item бясплатная галасавая сувязь унутры кампаніі;
    \item магчымасць інтэграцыі галасавых сервісаў у
          бізнес-праграмы і бізнес-працэсы;
    \item багатыя магчымасці па запісу размоў і вядзенню статыстыкі.
\end{enumerate}

На 2019 год кольскасць абанентаў IP-тэлефаніі ў Беларусі складае
прыблізна 2,8 міліонаў абанентаў.

IPTV (Internet Protocol Television) --- гэта тэхналогія лічбавага
тэлебачання ў сетках перадачы даных па пратаколу IP.
Асноўныя перавагі IPTV заключаюцца ў наступным:
\begin{enumerate}
    \item выбар каналаў і тэлеперадач;
    \item магчымасць кіравання тэлевізійнага эфіру;
    \item наяўнасць HD і FullHD каналаў;
    \item таннае абсталяванне і простая настройка;
    \item наяўнасць дадатковых паслуг (відэапракат, інтэрактыўны прагляд
          і інш.)
\end{enumerate}

Галоўнымі характарыстыкамі сеткі пры перадачы аўдыё- і відэапатокаў
з'яўляецца прапускная здольнасць і якасці сервіса.
Каб забяспечыць добрую якасць паслуг для VoIP сервіса
неабходна забяспечыць затрымку пакетаў паміж маршрутызатарамі не больш за
10 мс і імавернасць страты не больш за 3\%, для IPTV сервіса --
затрымка пакетаў не больш за 150 мс і імавернасць страты не больш за 1\%,
адпаведна.

Тэма курсавога праекта з'яўляецца актуальнай, так як пабудова аптымальнай
IP-сеткі дазваляе эфектыўна пастаўляць разнастайныя IP-паслугі і
падтрымліваць параметры якасці паслуг на добрым узроўні.

Мэтай дадзенага курсавога праекта з'яўляецца праектаванне сеткі правайдара
IP-паслуг.
