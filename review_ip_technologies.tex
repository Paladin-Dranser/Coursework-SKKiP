\section{IP-сетка. Архітэктура і паслугі}

\subsection{Структурная арганізацыя інфакамунікацыйнай сеткі}

Сеткавая тапалогія --- гэта канфігурацыя графа, вяршыням якога
адпавядаюць канцавыя вузлы сеткі (камп'ютар) і камунікацыйнае
абсталяванне (маршрутызатар), а рэбрамі --- фізічныя альбо
інфармацыйныя сувязі паміж вяршынямі.

Так як сетку асацыіруюць з графам, то пры яе аналізе выкарыстоўваюцца
такія тэрміны як <<вяршыня>>, <<рабро>>, <<маршрут>> і іншыя.

У тэорыі графаў вяршыняй называецца фундаментальная адзінка, якая
фарміруе графы, рабром --- лінія, якая злучае пару сумежных
вяршынь графа. Пад маршрутам разумеюць канечную паслядоўнасць адпаведных рэбраў, што злучаюць вяршыні \textit{i} і \textit{j}.

Адрозніваюць фізічную і лагічную тапалогію сеткі.
Лагічная і фізічная тапалогіі сеткі незалежныя адна ад адной.
Фізічная тапалогія --- гэта геаметрыя пабудовы сеткі, а лагічная тапалогія
вызначае напрамкі патокаў даных паміж вузламі сеткі і спосабы
перадачы даных.

На дадзены момант выкарыстоўваюцца наступныя віды фізічнай тапалогіі:
\begin{enumerate}
    \item шына;
    \item зорка;
    \item кола;
    \item кожны з кожным;
    \item змешаная.
\end{enumerate}

Сетка з шыннай тапалогіяй выкарыстоўвае адзін канал перадачы даных,
на канцы якога ўстанаўліваюцца канцавыя супраціўленні (тэрмінатары).
Даныя ад вузла сеткі перадаюцца ў абодва бакі. Тэрмінатары прадухіляюць
адбіванне сігналаў, гэта значыць выкарыстоўваюцца для гашэння сігналаў,
якія дасягаюць канцоў канала перадачы даных.
Такім чынам, інфармацыя паступае на ўсе вузлы, аднак прымаецца толькі
тым вузлом, якому яна прызначаная. У тапалогіі сеткі <<шіна>> асяроддзе
перадачы даных выкарыстоўваецца сумесна і адначасова ўсімі вузламі сеткі,
а сігналы ад вузлоў распаўсюджваюцца адначасова ва ўсе кірункі па
асяроддзю перадачы.

У сетцы пабудаванай па тапалогіі тыпу <<зорка>> кожны вузел
злучаецца кабелем да камутатара. Камутатар забяспечвае паралельнае
злучэнне вузлоў і, такім чынам, усе вузлы, падключаныя да сеткі,
могуць камунікаваць адзін з адным.
Даныя ад вузла сеткі перадаюцца праз камутатар па ўсіх лініях сувязі
ўсім вузлам. Інфармацыя паступае на вузлы, аднак прымаецца толькі
тымі станцыямі, якім яна прызначаная.

У сеткі з тапалогіяй <<кола>> ўсе вузлы злучаныя каналамі сувязі ў кола, па якому перадаюцца дадзеныя. Выхад аднаго вузла злучаецца з уваходам іншага. Пачаўшы рух з аднаго пункту, даныя, у канчатковым рахунку, трапляюць на яго пачатак. Дадзеныя ў коле заўсёды рухаюцца ў адным і тым жа кірунку.

У сеткі з тапалогіяй <<кожны з кожным>> кожны вузел злучаны з усімі
астатнімі вузламі. Галоўныя перавагі дадзенай тапалогіі: высокая
адмоваўстойлівасць і самая высокая прадукцыйнасці перадачы інфармацыі,
так як інфармацыя на любы вузел перадаецца напрамую. Галоўны недахоп:
грувасткасць і кошт, так як кожны вузел павінен мець вялікую колькасць
камунікацыйных партоў для злучэння з кожным вузлом і неабходна
вялікая колькасць кабеляў сувязі паміж вузламі.

У змешанай тапалогіі сеткі адначасова выкарыстоўваецца камбінацыя
з іншых тыпаў сеткавай тапалогіі.

На малюнку \ref{figure: Common Scheme} прадстаўлена
абагульненая схема IP-сеткі.

\subsection{Праграмнае забеспячэнне}

Для значнага павышэння прадукцыйнасці працы праекціроўшчыкаў і тым самым зніжэнне кошту распрацоўкі сетак сувязі неабходны сучасныя індустрыяльныя метады праектавання на базе шырокага выкарыстання сістэм аўтаматызаванага праектавання (САПР) сетак сувязі, якія ўключаюць у сябе комплекс сродкаў праграмнага, інфармацыйнага, тэхнічнага і іншых відаў забеспячэння.
Складнікам праграмнага забеспячэння САПР сетак сувязі з'яўляюцца пакет прыкладных праграм (ППП) аптымізацыі і аналізу сетак сувязі. Пры добра наладжанай узаемадзеянні чалавек і ЭВМ здольныя сумесна вырашаць такія праблемы, якія не могуць быць вырашаны імі паасобку, пры гэтым на кожнага ўскладаюцца тыя аспекты праблемы, якія ён вырашае найлепшым спосабам.

Сёння ва ўмовах павелічэння выдаткаў на стварэнне праграмнага забеспячэння (ПЗ) і павышэння працаёмкасці гэтых работ распрацоўку прыкладной праграмы разглядаюць як комплекс навукова-даследчых і канструктарскіх работ.
Кошт ПЗ пачынае пераўзыходзіць кошт вылічальных сродкаў, на якіх яно
выкарыстоўваецца.

ПЗ, прызначанае для аптымізацыі і аналізу інфакамунікацыйных сетак (ІкС) арыентуецца на правядзенне шматварыянтнасці структурна-сеткавых разлікаў і ўяўляе сабой праграмныя комплексы шматмэтавага прызначэння.

Пакет прыкладных праграм павінeн быць зручным для шырокага колу праекціроўшчыкаў з розным узроўнем падрыхтоўкі. Выгляд задання зыходнай інфармацыі павінен быць зручным для шырокага круга праекціроўшчыкаў з розным узроўнем падрыхтоўкі. Выгляд задання зыходнай інфармацыі павінен быць просты і блізкі да выгляду, прынятага ў дадзенай праблемнай вобласці. Інфармацыя, неабходная для кіравання працэсам аптымізацыі і выбару значэння параметраў алгарытмаў, павінна быць зразумелая карыстальніку і мець як мага меншы аб'ём. Найбольш пераважным з'яўляецца рэжым аўтаматычнага выбару наладкавых параметраў. Выходная інфармацыя павінна ўключаць інтэгральныя і дыферэнцыйныя характарыстыкі аптымальнага праекта, значэння кіраваных зменных у экстрэмальных пунктах Х*, а таксама характарыстыкі кіравання працэсам аптымізацыі і траекторыю руху пункту аптымізацыі. Складаная структура мэтавай функцыі і функцый-абмежаванняў ставіць асабліва жорсткія патрабаванні да часу пошуку экстрэмуму. Пажадана, каб самастойныя, але надзвычай працаёмкія задачы пошуку стартавай тапалогіі ўскладаліся на ППП.

У цэлым ПЗ павінна характарызавацца высокай надзейнасцю, эфектыўнасцю пошуку і магчымасцю замены модуляў.
ПЗ разліку структуры іерархічных ІкС з'яўляецца праграмнай рэалізацыяй мадэлі і алгарытмаў. Пакет праграм характарызуецца іерархічнасцю і модульнай структурах, гібкасцю да перабудоў і дазваляе пры дапамозе замены адпаведных карт рабіць пераарыентацыю праграм.
Пакет уключае ў свой склад манітор, функцыянальную падсістэму (FS) і аптымізацыйных падсістэму (OS).

Манітор выконвае ўвод і друк выходных даных, выбар і запуск модуляў пакета ў адпаведнасці з зададзеным рэжымам працы, запуск OS і друк выходных вынікаў.

FS складаецца з дванаццаці праграмных модуляў, прызначаных для разліку эканамічных, структурных і імавернасна-часавых характарыстык працэсаў дастаўкі пакетаў і тэхнічнага абслугоўвання. FS працуе пад кіраваннем OS.

Пералічым асноўныя модулі FS, арыентаваныя на разлік:
\begin{enumerate}
    \item сярэдніх даўжынь каналаў розных прыступак іерархіі;
    \item дыяметра графа зонавай падсеткі;
    \item сярэдняй даўжыні маршруту;
    \item зонавых каэфіцыентаў замыкання;
    \item затрымак у трактах;
    \item верагоднасцей дастаўкі для тракта;
    \item эфектыўнасці тэхнічнага абслугоўвання;
    \item прыведзеных выдаткаў;
    \item значэнняў штрафной функцыі;
    \item значэнняў ўсіх абмежаванняў.
\end{enumerate}

Этапу настройкі пакета на канкрэтную задачу павінна папярэднічаць фармалізацыя задачы ў тэрмінах і значэннях, уласцівых гэтаму ПЗ. Пералік магчымых пастановачных альтэрнатыў вызначаецца крытэрыем аптымальнасці, класам структур аптымізацыі, складам сістэмы абмежаванняў, дысцыплінамі абслугоўвання чэргаў і іншым. Кожнаму паказчыку ў пакеце праграм адпавядае праграмны ключ.

Устаноўкай ключа ў тое ці іншае значэнне задаецца адпаведны рэжым.
Далейшыя дзеянні праекціроўшчыка зводзяцца да падрыхтоўкі зыходных дадзеных, задання пачатковых значэнняў параметраў аптымізацыйных алгарытмаў і стартавага пункту, запуску ПЗ і аналізу атрыманага развязвання.

Пакет праграм разліку іерархічных ІКС з'яўляецца развіццём аналагічнага пакета, прызначанага для разліку паасобных непрыярытэтных сетак сувязі. Характарыстыкі ПЗ:
\begin{enumerate}
    \item аб'ём памяці, якую займае ПЗ;
    \item працягласць аптымізацыі аднаго праекта ІкС;
    \item тып кіруючай сістэмы.
\end{enumerate}

Малы аб'ём памяці, якую займае ПЗ, тлумачыцца адсутнасцю матрычных формаў прадстаўлення інфармацыі, а высокая хуткадзейнасць праграм -- аналітычным (формульным) выглядам мадэлі ІкС і эфектыўнымі алгарытмамі, якія выкарыстоўваюць ідэі як пакаардынатнага, так і групавога спуску.

ПЗ выкарыстоўваецца ў задачах тапалагічнага праектавання сетак сувязі, вылучэння эфектыўных абласцей выкарыстання розных метадаў камутацыі, ацэнкі гранічна дасягальных вартасных і імавернасна-часавых характарыстык сеткі, ацэнкі ўстойлівасці рашэння да выходных умоў задачы, выяўлення <<вузкіх>> па прапускной здольнасці месцаў.

Практыка аптымізацыі шэрагу агульнадзяржаўных і ведамасных сетак сувязі паказала, што ў адрозненні ад традыцыйных переборных працэдур тапалагічнага праектавання ПЗ дазваляе праводзіць дэталізацыю грамадскіх патрабаванняў па затрымцы, кошту, верагоднасці дастаўкі (страт) і надзейнасці да прыватных патрабаванняў, што прад'яўляюцца да асобных падсетак, што павышае эфектыўнасць наступнага прымянення традыцыйных переборных алгарытмаў праектавання. ПЗ выключае неабходнасць прымянення дапаможных алгарытмаў генерацыі дапушчальных стартавых структур, а таксама забяспечвае аптымізацыю і аналіз ІкС практычна неабмежаванага маштабу.

\subsection{Патрабаванні IP-паслуг да якасці абслугоўвання}

Рэкамендацыі Y.1540 і Y.1541 вызначаюць сеткавыя характарыстыкі і нормы якасці абслугоўвання ў сетках IP. Асноўныя падыходы да вырашэння задачы забеспячэння гарантаванай якасці абслугоўвання ў IP сетках грунтуюцца на мадэлях інтэграваных і дыферэнцыраваных паслуг.

Якасць абслугоўвання (Quality of Service, QoS) з'яўляецца прадметам актыўных даследаванняў і стандартызацыі на працягу ўсёй гісторыі развіцця тэлекамунікацый. Істотны ўклад у развіццё розных аспектаў канцэпцыі QoS унёс Міжнародны саюз электрасувязі (МСЭ), уключаючы, у тым ліку, распрацоўку нормаў і патрабаванняў да паказчыкаў якасці абслугоўвання, стандартызацыю сеткавых механізмаў, якія забяспечваюць неабходныя паказчыкі QoS, а таксама фармулёўку асноватворных азначэнняў.

Пры недахопе рэсурсаў, які вядзе да павелічэння верагоднасці страт пакетаў і росту іх затрымак, для праграм рэальнага часу неабходныя паказчыкі якасці абслугоўвання не могуць быць забяспечаны. Перш за ўсё, гэта тлумачыцца асноўным прынцыпам функцыянавання IP-сетак - перадачай даных у дейтаграммнам рэжыме, г. зн. без устанаўлення злучэнняў і без кіравання. Са з'яўленнем новых праграм, асабліва рэальнага часу (інтэрактыўная перадача прамовы, відэатэлефанія і відэаканферэнцыі), пытанне аб гарантаванай якасці абслугоўвання ў сетках IP становіцца адным з найбольш складаных. Гэта тлумачыць, чаму якасць абслугоўвання ў сетках IP застаецца прадметам пастаяннай увагі МСЭ, Еўрапейскага інстытута тэлекамунікацыйных стандартаў (ETSI), Інжынернай рады Інтэрнэта (IETF) і іншых арганізацый стандартызацыі ў электрасувязі.

Рэкамендацыя МСЭ Y.1540 апісвае стандартныя сеткавыя характарыстыкі для перадачы пакетаў у сетках IP. Рэкамендацыя МСЭ Y.1541 вызначае нормы для параметраў, вызначаных у Y.1540, паміж двума межавымі сеткавымі інтэрфейсам - кропкамі падключэння канцавых тэрмінальных прылад. Акрамя таго, у гэтай рэкамендацыі спецыфіцыраваны шэсць класаў якасці абслугоўвання ў залежнасці ад праграм.

Гэтыя рэкамендацыі важныя для ўсіх удзельнікаў тэлекамунікацыйнага сцэнарыя -- аператараў і правайдараў, вытворцаў абсталявання і канчатковых карыстальнікаў. Сеткавыя аператары і правайдары будуць выкарыстоўваць іх пры планаванні, разгортванні і ацэнцы сетак IP у адпаведнасці з патрабаваннямі канчатковых карыстальнікаў да якасці абслугоўвання. Вытворцы будуць абапірацца на гэтыя рэкамендацыі пры стварэнні абсталявання, якое павінна адказваць спецыфікацыям сеткавых правайдараў. Нарэшце, канчатковыя карыстальнікі (у першую чаргу, карпаратыўныя) змогуць прымяніць рэкамендацыі Y.1540 і Y.1541 пры ацэнцы характарыстык IP-сетак з пазіцый адпаведнасці гэтых характарыстык патрабаванням спажыўцоў.

У Рэкамендацыі Y.1540 разглядаюцца наступныя сеткавыя характарыстыкі, як найбольш важныя па ступені іх уплыву на скразную якасць абслугоўвання (ад крыніцы да атрымальніка), якая ацэньваецца карыстальнікам:
\begin{enumerate}
    \item прадукцыйнасць сеткі;
    \item надзейнасць сеткі / сеткавых элементаў;
    \item затрымка;
    \item варыяцыя затрымкі (джиттер);
    \item страты пакетаў.
\end{enumerate}


