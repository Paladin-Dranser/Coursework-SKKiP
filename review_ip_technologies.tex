\section{IP-сетка. Архітэктура і паслугі}

\subsection{Структурная арганізацыя інфакамунікацыйнай сеткі}

Сеткавая тапалогія --- гэта канфігурацыя графа, вяршыням якога
адпавядаюць канцавыя вузлы сеткі (камп'ютар) і камунікацыйнае
абсталяванне (маршрутызатар), а рэбрамі --- фізічныя альбо
інфармацыйныя сувязі паміж вяршынямі.

Так як сетку асацыіруюць з графам, то пры яе аналізе выкарыстоўваюцца
такія тэрміны як <<вяршыня>>, <<рабро>>, <<маршрут>> і іншыя.

У тэорыі графаў вяршыняй называецца фундаментальная адзінка, якая
фарміруе графы, рабром --- лінія, якая злучае пару сумежных
вяршынь графа. Пад маршрутам разумеюць канечную паслядоўнасць адпаведных рэбраў, што злучаюць вяршыні \textit{i} і \textit{j}.

Адрозніваюць фізічную і лагічную тапалогію сеткі.
Лагічная і фізічная тапалогіі сеткі незалежныя адна ад адной.
Фізічная тапалогія --- гэта геаметрыя пабудовы сеткі, а лагічная тапалогія
вызначае напрамкі патокаў даных паміж вузламі сеткі і спосабы
перадачы даных.

На дадзены момант выкарыстоўваюцца наступныя віды фізічнай тапалогіі:
\begin{enumerate}
    \item шына;
    \item зорка;
    \item кола;
    \item кожны з кожным;
    \item змешаная.
\end{enumerate}

Сетка з шыннай тапалогіяй выкарыстоўвае адзін канал перадачы даных,
на канцы якога ўстанаўліваюцца канцавыя супраціўленні (тэрмінатары).
Даныя ад вузла сеткі перадаюцца ў абодва бакі. Тэрмінатары прадухіляюць
адбіванне сігналаў, гэта значыць выкарыстоўваюцца для гашэння сігналаў,
якія дасягаюць канцоў канала перадачы даных.
Такім чынам, інфармацыя паступае на ўсе вузлы, аднак прымаецца толькі
тым вузлом, якому яна прызначаная. У тапалогіі сеткі <<шіна>> асяроддзе
перадачы даных выкарыстоўваецца сумесна і адначасова ўсімі вузламі сеткі,
а сігналы ад вузлоў распаўсюджваюцца адначасова ва ўсе кірункі па
асяроддзю перадачы.

У сетцы пабудаванай па тапалогіі тыпу <<зорка>> кожны вузел
злучаецца кабелем да камутатара. Камутатар забяспечвае паралельнае
злучэнне вузлоў і, такім чынам, усе вузлы, падключаныя да сеткі,
могуць камунікаваць адзін з адным.
Даныя ад вузла сеткі перадаюцца праз камутатар па ўсіх лініях сувязі
ўсім вузлам. Інфармацыя паступае на вузлы, аднак прымаецца толькі
тымі станцыямі, якім яна прызначаная.

У сеткі з тапалогіяй <<кола>> ўсе вузлы злучаныя каналамі сувязі ў кола, па якому перадаюцца дадзеныя. Выхад аднаго вузла злучаецца з уваходам іншага. Пачаўшы рух з аднаго пункту, даныя, у канчатковым рахунку, трапляюць на яго пачатак. Дадзеныя ў коле заўсёды рухаюцца ў адным і тым жа кірунку.

У змешанай тапалогіі сеткі адначасова выкарыстоўваецца камбінацыя
з іншых тыпаў сеткавай тапалогіі.
