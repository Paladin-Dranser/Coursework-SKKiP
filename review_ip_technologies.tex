\section{IP-сетка. Архітэктура і паслугі}

\subsection{Структурная арганізацыя інфакамунікацыйнай сеткі}

Сеткавая тапалогія --- гэта канфігурацыя графа, вяршыням якога
адпавядаюць канцавыя вузлы сеткі (камп'ютар) і камунікацыйнае
абсталяванне (маршрутызатар), а рэбрамі --- фізічныя альбо
інфармацыйныя сувязі паміж вяршынямі.

Так як сетку асацыіруюць з графам, то пры яе аналізе выкарыстоўваюцца
такія тэрміны як <<вяршыня>>, <<рабро>>, <<маршрут>> і іншыя.

У тэорыі графаў вяршыняй называецца фундаментальная адзінка, якая
фарміруе графы, рабром --- лінія, якая злучае пару сумежных
вяршынь графа. Пад маршрутам разумеюць канечную паслядоўнасць адпаведных рэбраў, што злучаюць вяршыні \textit{i} і \textit{j}.

Адрозніваюць фізічную і лагічную тапалогію сеткі.
Лагічная і фізічная тапалогіі сеткі незалежныя адна ад адной.
Фізічная тапалогія --- гэта геаметрыя пабудовы сеткі, а лагічная тапалогія
вызначае напрамкі патокаў даных паміж вузламі сеткі і спосабы
перадачы даных.

На дадзены момант выкарыстоўваюцца наступныя віды фізічнай тапалогіі:
\begin{enumerate}
    \item шына;
    \item зорка;
    \item кола;
    \item кожны з кожным;
    \item змешаная.
\end{enumerate}

Сетка з шыннай тапалогіяй выкарыстоўвае адзін канал перадачы даных,
на канцы якога ўстанаўліваюцца канцавыя супраціўленні (тэрмінатары).
Даныя ад вузла сеткі перадаюцца ў абодва бакі. Тэрмінатары прадухіляюць
адбіванне сігналаў, гэта значыць выкарыстоўваюцца для гашэння сігналаў,
якія дасягаюць канцоў канала перадачы даных.
Такім чынам, інфармацыя паступае на ўсе вузлы, аднак прымаецца толькі
тым вузлом, якому яна прызначаная. У тапалогіі сеткі <<шіна>> асяроддзе
перадачы даных выкарыстоўваецца сумесна і адначасова ўсімі вузламі сеткі,
а сігналы ад вузлоў распаўсюджваюцца адначасова ва ўсе кірункі па
асяроддзю перадачы.

У сетцы пабудаванай па тапалогіі тыпу <<зорка>> кожны вузел
злучаецца кабелем да камутатара. Камутатар забяспечвае паралельнае
злучэнне вузлоў і, такім чынам, усе вузлы, падключаныя да сеткі,
могуць камунікаваць адзін з адным.
Даныя ад вузла сеткі перадаюцца праз камутатар па ўсіх лініях сувязі
ўсім вузлам. Інфармацыя паступае на вузлы, аднак прымаецца толькі
тымі станцыямі, якім яна прызначаная.

У сеткі з тапалогіяй <<кола>> ўсе вузлы злучаныя каналамі сувязі ў кола, па якому перадаюцца дадзеныя. Выхад аднаго вузла злучаецца з уваходам іншага. Пачаўшы рух з аднаго пункту, даныя, у канчатковым рахунку, трапляюць на яго пачатак. Дадзеныя ў коле заўсёды рухаюцца ў адным і тым жа кірунку.

У сеткі з тапалогіяй <<кожны з кожным>> кожны вузел злучаны з усімі
астатнімі вузламі. Галоўныя перавагі дадзенай тапалогіі: высокая
адмоваўстойлівасць і самая высокая прадукцыйнасці перадачы інфармацыі,
так як інфармацыя на любы вузел перадаецца напрамую. Галоўны недахоп:
грувасткасць і кошт, так як кожны вузел павінен мець вялікую колькасць
камунікацыйных партоў для злучэння з кожным вузлом і неабходна
вялікая колькасць кабеляў сувязі паміж вузламі.

У змешанай тапалогіі сеткі адначасова выкарыстоўваецца камбінацыя
з іншых тыпаў сеткавай тапалогіі.

На малюнку \ref{figure: Common Scheme} прадстаўлена
абагульненая схема IP-сеткі.

\subsection{Праграмнае забеспячэнне}

Для значнага павышэння прадукцыйнасці працы праекціроўшчыкаў і тым самым зніжэнне кошту распрацоўкі сетак сувязі неабходны сучасныя індустрыяльныя метады праектавання на базе шырокага выкарыстання сістэм аўтаматызаванага праектавання (САПР) сетак сувязі, якія ўключаюць у сябе комплекс сродкаў праграмнага, інфармацыйнага, тэхнічнага і іншых відаў забеспячэння.
Складнікам праграмнага забеспячэння САПР сетак сувязі з'яўляюцца пакет прыкладных праграм (ППП) аптымізацыі і аналізу сетак сувязі. Пры добра наладжанай узаемадзеянні чалавек і ЭВМ здольныя сумесна вырашаць такія праблемы, якія не могуць быць вырашаны імі паасобку, пры гэтым на кожнага ўскладаюцца тыя аспекты праблемы, якія ён вырашае найлепшым спосабам.

Сёння ва ўмовах павелічэння выдаткаў на стварэнне праграмнага забеспячэння (ПЗ) і павышэння працаёмкасці гэтых работ распрацоўку прыкладной праграмы разглядаюць як комплекс навукова-даследчых і канструктарскіх работ.
Кошт ПЗ пачынае пераўзыходзіць кошт вылічальных сродкаў, на якіх яно
выкарыстоўваецца.

ПЗ, прызначанае для аптымізацыі і аналізу інфакамунікацыйных сетак (ІкС) арыентуецца на правядзенне шматварыянтнасці структурна-сеткавых разлікаў і ўяўляе сабой праграмныя комплексы шматмэтавага прызначэння.

Пакет прыкладных праграм павінeн быць зручным для шырокага колу праекціроўшчыкаў з розным узроўнем падрыхтоўкі. Выгляд задання зыходнай інфармацыі павінен быць зручным для шырокага круга праекціроўшчыкаў з розным узроўнем падрыхтоўкі. Выгляд задання зыходнай інфармацыі павінен быць просты і блізкі да выгляду, прынятага ў дадзенай праблемнай вобласці. Інфармацыя, неабходная для кіравання працэсам аптымізацыі і выбару значэння параметраў алгарытмаў, павінна быць зразумелая карыстальніку і мець як мага меншы аб'ём. Найбольш пераважным з'яўляецца рэжым аўтаматычнага выбару наладкавых параметраў. Выходная інфармацыя павінна ўключаць інтэгральныя і дыферэнцыйныя характарыстыкі аптымальнага праекта, значэння кіраваных зменных у экстрэмальных пунктах Х*, а таксама характарыстыкі кіравання працэсам аптымізацыі і траекторыю руху пункту аптымізацыі. Складаная структура мэтавай функцыі і функцый-абмежаванняў ставіць асабліва жорсткія патрабаванні да часу пошуку экстрэмуму. Пажадана, каб самастойныя, але надзвычай працаёмкія задачы пошуку стартавай тапалогіі ўскладаліся на ППП.

У цэлым ПЗ павінна характарызавацца высокай надзейнасцю, эфектыўнасцю пошуку і магчымасцю замены модуляў.
ПЗ разліку структуры іерархічных ІкС з'яўляецца праграмнай рэалізацыяй мадэлі і алгарытмаў. Пакет праграм характарызуецца іерархічнасцю і модульнай структурах, гібкасцю да перабудоў і дазваляе пры дапамозе замены адпаведных карт рабіць пераарыентацыю праграм.
Пакет уключае ў свой склад манітор, функцыянальную падсістэму (FS) і аптымізацыйных падсістэму (OS).

Манітор выконвае ўвод і друк выходных даных, выбар і запуск модуляў пакета ў адпаведнасці з зададзеным рэжымам працы, запуск OS і друк выходных вынікаў.

FS складаецца з дванаццаці праграмных модуляў, прызначаных для разліку эканамічных, структурных і імавернасна-часавых характарыстык працэсаў дастаўкі пакетаў і тэхнічнага абслугоўвання. FS працуе пад кіраваннем OS.

Пералічым асноўныя модулі FS, арыентаваныя на разлік:
\begin{enumerate}
    \item сярэдніх даўжынь каналаў розных прыступак іерархіі;
    \item дыяметра графа зонавай падсеткі;
    \item сярэдняй даўжыні маршруту;
    \item зонавых каэфіцыентаў замыкання;
    \item затрымак у трактах;
    \item верагоднасцей дастаўкі для тракта;
    \item эфектыўнасці тэхнічнага абслугоўвання;
    \item прыведзеных выдаткаў;
    \item значэнняў штрафной функцыі;
    \item значэнняў ўсіх абмежаванняў.
\end{enumerate}

Этапу настройкі пакета на канкрэтную задачу павінна папярэднічаць фармалізацыя задачы ў тэрмінах і значэннях, уласцівых гэтаму ПЗ. Пералік магчымых пастановачных альтэрнатыў вызначаецца крытэрыем аптымальнасці, класам структур аптымізацыі, складам сістэмы абмежаванняў, дысцыплінамі абслугоўвання чэргаў і іншым. Кожнаму паказчыку ў пакеце праграм адпавядае праграмны ключ.

Устаноўкай ключа ў тое ці іншае значэнне задаецца адпаведны рэжым.
Далейшыя дзеянні праекціроўшчыка зводзяцца да падрыхтоўкі зыходных дадзеных, задання пачатковых значэнняў параметраў аптымізацыйных алгарытмаў і стартавага пункту, запуску ПЗ і аналізу атрыманага развязвання.

Пакет праграм разліку іерархічных ІКС з'яўляецца развіццём аналагічнага пакета, прызначанага для разліку паасобных непрыярытэтных сетак сувязі. Характарыстыкі ПЗ:
\begin{enumerate}
    \item аб'ём памяці, якую займае ПЗ;
    \item працягласць аптымізацыі аднаго праекта ІкС;
    \item тып кіруючай сістэмы.
\end{enumerate}

Малы аб'ём памяці, якую займае ПЗ, тлумачыцца адсутнасцю матрычных формаў прадстаўлення інфармацыі, а высокая хуткадзейнасць праграм -- аналітычным (формульным) выглядам мадэлі ІкС і эфектыўнымі алгарытмамі, якія выкарыстоўваюць ідэі як пакаардынатнага, так і групавога спуску.

ПЗ выкарыстоўваецца ў задачах тапалагічнага праектавання сетак сувязі, вылучэння эфектыўных абласцей выкарыстання розных метадаў камутацыі, ацэнкі гранічна дасягальных вартасных і імавернасна-часавых характарыстык сеткі, ацэнкі ўстойлівасці рашэння да выходных умоў задачы, выяўлення <<вузкіх>> па прапускной здольнасці месцаў.

Практыка аптымізацыі шэрагу агульнадзяржаўных і ведамасных сетак сувязі паказала, што ў адрозненні ад традыцыйных переборных працэдур тапалагічнага праектавання ПЗ дазваляе праводзіць дэталізацыю грамадскіх патрабаванняў па затрымцы, кошту, верагоднасці дастаўкі (страт) і надзейнасці да прыватных патрабаванняў, што прад'яўляюцца да асобных падсетак, што павышае эфектыўнасць наступнага прымянення традыцыйных переборных алгарытмаў праектавання. ПЗ выключае неабходнасць прымянення дапаможных алгарытмаў генерацыі дапушчальных стартавых структур, а таксама забяспечвае аптымізацыю і аналіз ІкС практычна неабмежаванага маштабу.

\subsection{Патрабаванні IP-паслуг да якасці абслугоўвання}

Рэкамендацыі Y.1540 і Y.1541 вызначаюць сеткавыя характарыстыкі і нормы якасці абслугоўвання ў сетках IP. Асноўныя падыходы да вырашэння задачы забеспячэння гарантаванай якасці абслугоўвання ў IP сетках грунтуюцца на мадэлях інтэграваных і дыферэнцыраваных паслуг.

Якасць абслугоўвання (Quality of Service, QoS) з'яўляецца прадметам актыўных даследаванняў і стандартызацыі на працягу ўсёй гісторыі развіцця тэлекамунікацый. Істотны ўклад у развіццё розных аспектаў канцэпцыі QoS унёс Міжнародны саюз электрасувязі (МСЭ), уключаючы, у тым ліку, распрацоўку нормаў і патрабаванняў да паказчыкаў якасці абслугоўвання, стандартызацыю сеткавых механізмаў, якія забяспечваюць неабходныя паказчыкі QoS, а таксама фармулёўку асноватворных азначэнняў.

Пры недахопе рэсурсаў, які вядзе да павелічэння верагоднасці страт пакетаў і росту іх затрымак, для праграм рэальнага часу неабходныя паказчыкі якасці абслугоўвання не могуць быць забяспечаны. Перш за ўсё, гэта тлумачыцца асноўным прынцыпам функцыянавання IP-сетак - перадачай даных у дейтаграммнам рэжыме, г. зн. без устанаўлення злучэнняў і без кіравання. Са з'яўленнем новых праграм, асабліва рэальнага часу (інтэрактыўная перадача прамовы, відэатэлефанія і відэаканферэнцыі), пытанне аб гарантаванай якасці абслугоўвання ў сетках IP становіцца адным з найбольш складаных. Гэта тлумачыць, чаму якасць абслугоўвання ў сетках IP застаецца прадметам пастаяннай увагі МСЭ, Еўрапейскага інстытута тэлекамунікацыйных стандартаў (ETSI), Інжынернай рады Інтэрнэта (IETF) і іншых арганізацый стандартызацыі ў электрасувязі.

Рэкамендацыя МСЭ Y.1540 апісвае стандартныя сеткавыя характарыстыкі для перадачы пакетаў у сетках IP. Рэкамендацыя МСЭ Y.1541 вызначае нормы для параметраў, вызначаных у Y.1540, паміж двума межавымі сеткавымі інтэрфейсам - кропкамі падключэння канцавых тэрмінальных прылад. Акрамя таго, у гэтай рэкамендацыі спецыфіцыраваны шэсць класаў якасці абслугоўвання ў залежнасці ад праграм.

Гэтыя рэкамендацыі важныя для ўсіх удзельнікаў тэлекамунікацыйнага сцэнарыя -- аператараў і правайдараў, вытворцаў абсталявання і канчатковых карыстальнікаў. Сеткавыя аператары і правайдары будуць выкарыстоўваць іх пры планаванні, разгортванні і ацэнцы сетак IP у адпаведнасці з патрабаваннямі канчатковых карыстальнікаў да якасці абслугоўвання. Вытворцы будуць абапірацца на гэтыя рэкамендацыі пры стварэнні абсталявання, якое павінна адказваць спецыфікацыям сеткавых правайдараў. Нарэшце, канчатковыя карыстальнікі (у першую чаргу, карпаратыўныя) змогуць прымяніць рэкамендацыі Y.1540 і Y.1541 пры ацэнцы характарыстык IP-сетак з пазіцый адпаведнасці гэтых характарыстык патрабаванням спажыўцоў.

У Рэкамендацыі Y.1540 разглядаюцца наступныя сеткавыя характарыстыкі, як найбольш важныя па ступені іх уплыву на скразную якасць абслугоўвання (ад крыніцы да атрымальніка), якая ацэньваецца карыстальнікам:
\begin{enumerate}
    \item прадукцыйнасць сеткі;
    \item надзейнасць сеткі і сеткавых элементаў;
    \item затрымка;
    \item варыяцыя затрымкі (джытэр);
    \item страты пакетаў.
\end{enumerate}

Прадукцыйнасць сеткі (або хуткасць перадачы даных) карыстальніка вызначаецца як эфектыўная хуткасць перадачы, якая вымяраецца ў бітах у секунду. Варта адзначыць, што значэнне гэтага параметру не супадае з максімальнай прапускной здольнасцю сеткі. Мінімальнае значэнне прадукцыйнасці звычайна гарантуецца правайдарам паслуг, які, у сваю чаргу, павінен мець адпаведныя гарантыі ад сеткавага правайдара.

Карыстальнікі звычайна чакаюць высокі ўзровень надзейнасці ад сістэм сувязі. Надзейнасць сеткі можа быць вызначана праз шэраг параметраў, з якіх найбольш часта выкарыстоўваецца каэфіцыент гатоўнасці, вылічваецца як адносіны часу прастою аб'екта да сумарнга часу назірання аб'екта, які ўключае час прастою і час паміж адмовамі. У ідэальным выпадку каэфіцыент гатоўнасці павінен быць роўны 1, што азначае стоадсоткавую гатоўнасць сеткі. На практыцы каэфіцыент гатоўнасці ацэньваецца лікам "дзявятак". Напрыклад "тры дзявяткі" азначаюць, што каэфіцыент гатовасці складае 0,999, што адпавядае 9 гадзінам часу недаступнасці (прастою) сеткі ў год.

Затрымка дастаўкі пакета IP (IP packet transfer delay, IPTD) вызначаецца як час паміж дзвюма падзеямі ($t_2 - t_1$) -- уводам пакета ва ўваходны пункт сеткі ў момант $t_1$ і выхадам пакета з выходнага пункту сеткі ў момант $t_2$, дзе $t_2 > t_1$ і $t_2 - t_1 \le T_\text{mах}$. Увогуле, параметр IPTD вызначаецца як час дастаўкі пакета паміж крыніцай і атрымальнікам для ўсіх пакетаў -- як паспяхова перададзеных, так і з памылкамі.

Сярэдняя затрымка дастаўкі пакета IP -- параметр, спецыфіцыраваны ў рэкамендацыі Y.1540, вызначаецца як сярэдняя арыфметычная велічыня затрымак пакетаў ў абраным наборы перададзеных і прынятых пакетаў. Значэнне сярэдняй затрымкі залежыць ад перадаванага ў сетцы трафіку і даступных сеткавых рэсурсаў, у прыватнасці, ад прапускной здольнасці. Рост нагрузкі і памяншэнне даступных сеткавых рэсурсаў вядуць да росту чэргаў у вузлах сеткі і, як следства, да павелічэння сярэдніх затрымак дастаўкі пакетаў.

Моўная інфармацыя і відэаінфармацыя з'яўляюцца прыкладамі трафіку, адчувальнага да затрымак, тады як перадача тэкставаных даных у асноўным менш адчувальная да затрымак. Калі затрымка дастаўкі пакета перавышае вызначаныя значэнні $T_\text{mах}$, такія пакеты адкідаюцца. У праграмах рэальнага часу (напрыклад, у IP-тэлефаніі) гэта вядзе да пагаршэння якасці гаворкі. Абмежаванні, звязаныя з сярэдняй затрымкай пакетаў IP, гуляюць ключавую ролю для паспяховага ўкаранення тэхналогіі Voice over IP (VoIP), відэа-канферэнцый і іншых прыкладанняў рэальнага часу.

Варыяцыя затрымкі пакета IP (IP packet delay variation, IPDV). Параметр $V_k$, характарызуе варыяцыю затрымкі IPDV. Для IP-пакета з індэксам k гэты параметр вызначаецца паміж ўваходным і выходным пунктамі сеткі ў выглядзе рознасці паміж абсалютнай велічынёй затрымкі $X_k$ пры дастаўцы пакета з індэксам k, і пэўнай эталоннай (або апорнай) велічынёй затрымкі дастаўкі пакета IP, $d1{,}2$, для тых жа сеткавых пунктаў: $V_k = X_k - d1{,}2$.

Эталонная затрымка дастаўкі пакета IP, $d1{,}2$, паміж крыніцай і атрымальнікам вызначаецца як абсалютнае значэнне затрымкі дастаўкі першага пакета IP паміж дадзенымі сеткавымі пунктамі. Варыяцыя затрымкі пакета IP, або джытэр, выяўляецца ў тым, што паслядоўныя пакеты прыбываюць да атрымальніка ў нерэгулярныя моманты часу. У сістэмах IP-тэлефаніі гэта, да прыкладу, вядзе да скажэнняў гуку і ў выніку чаго гаворка становіцца непераборлівай.

Каэфіцыент страты пакетаў IP (IP packet loss ratio, IPLR) вызначаецца як стаўленне сумарнай колькасці страчаных пакетаў да агульнай колькасці прынятых у абраным наборы перададзеных і прынятых пакетаў. Страты пакетаў у сетках IP ўзнікаюць у тым выпадку, калі значэнне затрымак пры іх перадачы перавышае нармаванае значэнне, вызначанае вышэй як $T_\text{mах}$. Калі пакеты губляюцца, то пры перадачы дадзеных магчыма іх паўторная перадача па запыце прымаючага боку. У сістэмах VoIP пакеты, якія прыйшлі да атрымальніка з затрымкай, якая перавышае $T_\text{mах}$, адкідаюцца, што вядзе да правалаў у прыманай гаворкі. Сярод прычын, якія выклікаюць страты пакетаў, неабходна адзначыць рост чэргаў у вузлах сеткі, якія ўзнікаюць пры перагрузках.

Каэфіцыент памылак пакетаў IP (IP packet error ratio, IPER) вызначаецца як сумарная колькасць пакетаў, прынятых з памылкамі, да сумы паспяхова прынятых і пакетаў, прынятых з памылкамі.
Рэкамендацыя Y.1541 ўстанаўлівае адпаведнасць паміж класамі якасці абслугоўвання:
\begin{enumerate}
    \item Клас 0 -- праграмы рэальнага часу, адчувальныя да джытэру, характарызаваныя высокім узроўнем інтэрактыўнасці (VoIP, відэаканферэнцыі);
    \item Клас 1 -- праграмы рэальнага часу, адчувальныя да джытэру, інтэрактыўныя (VoIP, відэаканферэнцыі);
    \item Клас 2 -- транзакцыі даных, характарызаваныя высокім узроўнем інтэрактыўнасці (напрыклад, сігналізацыя);
    \item Клас 3 -- транзакцыі даных, інтэрактыўныя;
    \item Клас 4 -- праграмы, якія дапускаюць нізкі ўзровень страт (кароткія транзакцыі, масівы даных);
    \item Клас 5 -- традыцыйныe прымяненнe сетак IP.
\end{enumerate}

\newpage

У табліцы \ref{table: Class Charactaristics} прадстаўлены нормы сеткавыя характарыстыкі для розных класаў.

\begin{table}[h!]
    \caption{Нормы для характарыстык IP сетак з
             размеркаваннем па класахх якасці абслугоўвання}
    \begin{tabularx}{\textwidth}{|>{\centering\arraybackslash}X|
                                 c|c|c|c|c|c|}
        \hline
        \multirow{2}{*}{Сеткавыя характарыстыкі} & \multicolumn{6}{c|}{Класы QoS} \\
        \cline{2-7}
                  & 0 & 1 & 2 & 3 & 4 & 5 \\
        \hline
        Затрымка дастаўкі пакета IP, IPTD & 100 мс & 400 мс & 100 мс & 400 мс & 1 с & ---  \\
        \hline
        Варыяцыя затрымкі пакета IP, IPDV & 50 мс & 50 мс & --- & --- & --- & --- \\
        \hline
        Каэфіцыент страты пакетаў IP, IPLR & 0,001 & 0,001 & 0,001 & 0,001 & 0,001 & --- \\
        \hline
        Каэфіцыент памылакпакетаў IP, IPER & 0,0001 & 0,0001 & 0,0001 & 0,0001 & 0,0001 & --- \\
        \hline
    \end{tabularx}
    \label{table: Class Charactaristics}
\end{table}

\vspace{-\baselineskip}
\subsubsection{VoIP кодэкі.}

Аўдыякодэк -- гэта камп'ютарная праграма або апаратны сродак, прызначаны для кадавання або дэкадавання аудыёданых. У IP-тэлефаніі на сённяшні дзень найбольш распаўсюджана пераўтварэнне з дапамогай кодэка G.729.

Кодэкі характарызуюцца наступнымі параметрамі:
\begin{enumerate}
    \item паласой прапускання лічбавага канала, гэта значыць максімальнай хуткасцю перадачы інфармацыі па дадзеным канале ў бітах;
    \item наяўнасць тэхналогіі скарачэння інфармацыі, якая перадаецца ў перыяды маўчання (VAD), і функцыі генерацыі камфортнага шуму (CNG);
    \item памер кадра -- гэта мінімальная, тэарэтычна вызначаная метадам кадавання, затрымка перадачы інфармацыі;
    \item адчувальнасць да страты кадраў.
\end{enumerate}

Ўсе існуючыя сёння тыпы моўных кодэкаў, якія ўжываюцца ў iP-тэлефаніі, па прынцыпе дзеяння можна падзяліць на тры групы:
\begin{enumerate}
    \item кодэк з імпульсна кодавай мадуляцыяй (ІКМ) і адаптыўнай дыферэнцыяльнай імпульсна кодавай мадуляцыяй (АДІКМ), якія выкарыстоўваюцца сёння ў сістэмах традыцыйнай тэлефаніі. У большасці выпадкаў, ўяўляюць сабой спалучэнне АЛП / ЛАП;
    \item кодэк з вакодэрным пераўтварэннем моўнага сігналу паўсталі ў сістэмах мабільнай сувязі для зніжэння патрабаванняў да прапускной здольнасці радыё тракту. Гэтая група кодэкаў выкарыстоўвае гарманічны сінтэз сігналу на падставе інфармацыі аб яго вакальных складнікаў - фанемы. У большасці выпадкаў, такія кодэкі рэалізаваны як аналагавыя прылады;
    \item Камбінаваныя (гібрыдныя) кодэкі спалучаюць у сабе тэхналогію вакодэрнага пераўтварэння/сінтэзу гаворкі, але аперуюць ўжо з лічбавым сігналам з дапамогай спецыялізаваных прылад. Кодэкі гэтага тыпу ўтрымліваюць у сабе ІКМ або АДІКМ кодэк і рэалізаваны лічбавым спосабам вакодар.
\end{enumerate}

Першыя ІКМ кодэкі з нелінейным квантаваннeм з'явіліся ўжо ў 60-х гг. Кодэк G.711 шырока распаўсюджаны ў сістэмах традыцыйнай тэлефаніі з камутацыяй каналаў. Нягледзячы на тое, што рэкамендацыя G.711 ў стандарце Н.323 з'яўляецца асноўнай і першаснай, у шлюзах IP-тэлефаніі дадзены кодэк прымяняецца рэдка з-за высокіх патрабаванняў да паласы прапускання і затрымак ў канале перадачы. Выкарыстанне G.711 ў сістэмах IP-тэлефаніі абгрунтавана толькі ў тых выпадках, калі патрабуецца забяспечыць максімальную якасць кадавання маўленчай інфармацыі пры невялікім ліку адначасовых размоў.

Рэкамендацыя G.723.1 апісвае гібрыдныя кодэкі, якія выкарыстоўваюць тэхналогію кадавання моўнай інфармацыі, скарочана званую як MP-MLQ (MultyPulse-MultyLevel Quan\-ti\-za\-tion -- множная імпульсная шматузроўневая квантызацыя), дадзеныя кодэкі можна ахарактарызаваць, як камбінацыю АЛП/ЛАП і вакодэр. Як ужо згадвалася вышэй, сваім узнікненнем гібрыдныя кодэкі абавязаны сістэмам мабільнай сувязі. Прымяненне вакодэра дазваляе знізіць хуткасць перадачы дадзеных у канале, што прынцыпова важна для эфектыўнага выкарыстання як радыётракта, так і IP-канала. Асноўны прынцып працы вакодэра -- сінтэз зыходнага моўнага сігналу з дапамогай адаптыўнай замены яго гарманічных складнікаў адпаведным наборам частотных фанем і ўзгодненымі шумавымі каэфіцыентамі. Кодэк G.723 ажыццяўляе пераўтварэнне аналагавага сігналу ў паток дадзеных з хуткасцю 64 Кбіт/с (ІКМ), а затым пры дапамозе шматпалоснага лічбавага фільтра/вакодэра вылучае частотныя фанемы, аналізуе іх і перадае па IP-каналу інфармацыю толькі аб бягучым стане фанем у моўным сігнале. Дадзены алгарытм пераўтварэння дазваляе знізіць хуткасць кадаванай інфармацыі да 5,3-6,3 Кбіт/с без заўважнага пагаршэння якасці гаворкі. Кодэк мае дзве хуткасці і два варыянты кадавання: 6,3 Кбіт/с з алгарытмам MP-MLQ і 5,3 Кбіт/с з алгарытмам CELP. Першы варыянт прызначаны для сетак з пакетнай перадачай голасу і забяспечвае лепшае якасць кадавання ў параўнанні з варыянтам CELP, але менш адаптаваны да выкарыстання ў сетках са змeшаным тыпам трафіку (голас/даныя)[\ref{copy-past: 5}].

Працэс пераўтварэння патрабуе ад DSP 16,4-16,7 MIPS (MillionInstructionsPerSecond) і ўносіць затрымку 37 мс. Кодэк G.723.1 шырока ўжываецца ў галасавых шлюзах і іншых прыладах IP-тэлефаніі. Кодэк саступае па якасці кадавання прамовы кодэку G.729а, але меней патрабавальны да рэсурсаў працэсара і прапускной здольнасці канала.

G.729 сямейства ўключае кодэкі G.729, G.729 Annex А, G.729 Annex B (змяшчае Voice Activity Detection і генератар камфортнага шуму). Кодэкі G.729 называюць Conjugate Structure-Algebraic Code Excited Linear Prediction (CS-ACELP) --- спалучаная структура з кіраваным алгебраічным кодам лінейным прадказаннем. Працэс пераўтварэння выкарыстоўвае 21,5 MIPS і ўносіць затрымку 15 мс. Хуткасць кадаванага маўленчага сігналу складае 8 Кбіт/с. У прыладах VoIP дадзены кодэк займае лідзіруючае месца, забяспечваючы найлепшую якасць кадавання моўнай інфармацыі пры досыць высокай кампрэсіі.

Рэкамендацыя G.726 апісвае тэхналогію кадавання з выкарыстаннем адаптыўнай дыферэнцыйнай імпульсна-кодавай мадуляцыі (АДІКМ) з хуткасцямі: 32 кбіт/с, 24 кбіт/с, 16 кбіт/с. Працэс пераўтварэння не ўносіць істотнай затрымкі і патрабуе ад DSP 5,5-6,4 MIPS. Кодэк можа прымяняцца сумесна з кодэкам G.711 для паніжэння хуткасці кадавання апошняга. Кодэк прызначаны для выкарыстання ў сістэмах відэаканфeрэнцый. Гібрыдны кодэк, апісаны ў рэкамендацыі G.728 ў 1992 г, адносіцца да катэгорыі LD-CELP (Low Delay Code Excited Linear Prediction). Кодэк з кіраваным кодам лінейным прадказаннем і малой затрымкай. Кодэк забяспечвае хуткасць пераўтварэнні 16 Кбіт/с, уносіць затрымку пры кадаванні ад 3 да 5 мс і прызначаны для выкарыстання ў сістэмах відэаканфeрэнцый. У прыладах IP-тэлефаніі дадзены кодэк ўжываец\-ца досыць рэдка.

У табліцы \ref{copy-past-table: 1} прадстаўлены характарыстыкі кодэкаў.

\begin{table}[h!]
    \caption{Характарыстыкі кодэкаў}
    \begin{tabularx}{\textwidth}{|c|c|c|>{\centering\arraybackslash}X|}
        \hline
        Кодэк & Тып кодэка & Хуткасць кадавання, Кбіт/с & Затрымка пры кадаванні, мс \\
        \hline
        G.711 & ІКМ & 64 & 0,75 \\
        \hline
        G.726 & АДІКМ & 32 & 1 \\
        \hline
        G.728 & LD-CELP & 16 & 3-5 \\
        \hline
        G.729 & CS-ACELP & 8 & 10 \\
        \hline
        G.726a & CS-ACELP & 8 & 10 \\
        \hline
        G.723.1 & MP-MLQ & 6,3 & 30 \\
        \hline
        G.723.1 & ACELP & 5,3 & 30 \\
        \hline
    \end{tabularx}
    \label{copy-past-table: 1}
\end{table}

\vspace{-\baselineskip}
\subsubsection{IPTV кодэкі.}
Асноўныя метады кампрэсіі, якія прымяняюцца, можна падзяліць на дзве групы. Першую групу складаюць метады, якія выкарыстоўваюць унутракадравы сціск (кожны кадр апрацоўваецца асобна). Гэта JPEG (Motion JPEG), Wavelet і JPEG-2000. У другую групу ўваходзяць метады, якія выкарыстоўваюць міжкадравае альбо паточнае сцісканне: MPEG-1, H.261/H.263, MPEG-2, MPEG-4. Існуюць разнавіднасці Wavelet, у якіх ужываецца міжкадравы сціск (так званая дэльта-кампрэсія). Асноўныя характарыстыкі розных метадаў прыведзены ў табліцы
\ref{copy-past-table: 2}.

\begin{table}[h!]
    \caption{Патрабаванні да прапускной здольнасці (відэа са сцісканнем}
    \begin{tabularx}{\textwidth}{|c|>{\centering\arraybackslash}X|c|}
        \hline
        Фармат відэа & Прапускная здольнасць, Мбіт/с & Тып распазнавання \\
        \hline
        Motion JPEG & 5-16 & QCIF, CIF, 4CIF \\
        \hline
        H.261 & 0,03-1,92 & QCIF, CIF \\
        \hline
        H.263 & 0,03-4 & QCIF, CIF, 4CIF, 16CIF \\
        \hline
        MPEG-4 & 0,1-100 & QCIF, CIF, 4 CIF, 16CIF \\
        \hline
        MPEG-1 & 0,4-2 & QCIF, CIF \\
        \hline
        MPEG-2 & 0,3-100 & QCIF, CIF, 4CIF, 16CIF \\
        \hline
        JPEG-2000 & 1,5-5 & QCIF, CIF, 4CIF, 16CIF \\
        \hline
    \end{tabularx}
    \label{copy-past-table: 2}
\end{table}

Пры выкарыстанні MPEG-2 як найбольш распаўсюджанага фармату лічбавага сціску відэаданых, кожны тэлевізійны канал займае ў IP-сеткі ад 3,5 да 6 Мбіт/с. Перадача абранага абанентам IP-сеткі тэлевізійнага канала рэалізуецца на базе тэхналогіі IP-multicast або для выпадку прагляду відэа па замове на базе IP-unicast.

Стандарт MPEG-4 быў істотна дапрацаваны ў параўнанні з MPEG-2 па некалькіх нап\-рам\-ках: дададзены новыя схемы разбіцця зыходнай відэакарцінкі на зыходныя блокі, пашырана схема пошуку вектараў руху і ўскладненыя алгарытмы кампенсацыі руху. Гэтым было дасягнута істотнае паляпшэнне якасці відэакарцінкі пры паточнай відэатрансляцыі. Але практычна ніякага паляпшэння не адбылося для рэжыму пакадравага запісу.

Ва ўсіх метадах кампрэсіі ёсць свае вартасці і недахопы.

MPEG-падобныя, аптымізаваныя па хуткасці алгарытмы кампрэсіі Н.261 і Н.263 (з мадыфікацыямі Н.261+, Н.263+) распрацоўваліся ў разліку на відэаканферэнцыі і відэатэлефаніі па сетках ISDN, гэта значыць на праграмы з практычнай адсутнасцю дынамічных сцэн. На дум\-ку большасці экспертаў, фармаваныя імі выявы (пры высокай дынаміцы) маюць блокавую структуру, з'яўляюцца невыразнымі. Па ступені кампрэсіі яны займаюць прамежкавае месца паміж Wavelet і MPEG і сустракаюцца ў лічбавых сістэмах відэазапіса даволі рэдка.

MPEG-1 забяспечвае якасць, прыблізна эквівалентную VHS, і можа выкарыстоўвацца ў выпадку, калі прапускная здольнасць канала абмежаваная 2 Мбіт/с. У той жа час якасць перадачы дынамічных сцэн, характэрных для сістэмы відэанагляду за рухам транспарту, даволі невысокая.

MPEG-2 распрацоўваўся ў разліку на прапускную здольнасць канала перадачы 0,3-100 Мбіт/с з арыентацыяй на тэлебачанне высокай выразнасці (HDTV). Пры каэфіцыентах сціску да 30:1 ён забяспечвае якасць, прыблізна эквівалентную DVD, з добрым прайграваннем дынамічных сцэн. У той жа час пры вялікіх каэфіцыентах сціску (да 200:1) пачынаюць праяўляцца эфекты мазаікі, скажэнні. Гібкасць алгарытму дазваляе выкарыстоўваць яго ў розных сферах: ад здымкі і мантажу відэапрадукцыі да наступнага яе лічбавага распаўсюду (DVD) і перадачы (Digital Video Broadcasting). Відавочнае адрозненне гэтых задач прыводзіць да разнастайнасці магчымых схем кадавання. Моцны бок MPEG-2 -- строга вызначаная міжнародным стандартам працэдура дэкадавання. Гэта азначае, што нягледзячы на магчымыя адрозненні выкарыстаных схем кампрэсіі, любой MPEG-2 паток будзе паспяхова прайграны стандартным MPEG-2 дэкодэрам. Такім чынам, гарантуецца сумяшчальнасць абсталявання розных пастаўшчыкоў.

MPEG-3 распрацоўваўся як стандарт кадавання аўдыё і відэа для тэлебачання высокай выразнасці, які мае хуткасць перадачы даных у дыяпазоне ад 20 да 40 Mбіт/с.

MPEG-4 з'яўляецца стандартам, арыентаваным на кліент-серверныя сістэмы дастаўкі інтэрактыўнага мультымедыя па сетцы. Аўдыёвізуальныя сцэны ў ім разглядаюцца як аб'екты, якія мультыплексуюцца ў адзіны паток нароўні з дадатковым апісаннем сцэн. Гэта дае магчымасць карыстальніку пры прайграванні самому кіраваць працэсам прэзентацыі. У некаторых адносінах (у прыватнасці, па якасці відэа) ён з'яўляецца крокам назад у параўнанні з MPEG-2, аднак пры той жа біт-хуткасці, што і Н.261, ён забяспечвае істотна (на 30-40\%) больш высокую якасць. Для распазнавання 4CIF (Full) сярэдні інфармацыйны паток складае каля 1 Мбіт/с (лепш, чым у MPEG-2 і Wavelet).

Агульны недахоп метадаў кампрэсіі сямейства MPEG заключаецца ў тым, што яны прак\-тыч\-на перастаюць працаваць пры мультыплексаванні відэасігналаў, калі могуць узнікаць затрымкі паміж асобнымі відэакадрамі да 100-200 мс і больш, а пры інтэрвале паміж кадрамі больш 475 мс з'яўляецца супярэчнасць са стандартам MPEG-2 System, якое можа прывесці да немагчымасці прайгравання запісаў некаторымі дэкодэрамі. Другі істотны недахоп метадаў кам\-прэ\-сіі сямейства MPEG --- сінтэтычнасць <<нe апорных>> відэакадраў, што робіць іх аб\-са\-лют\-на непрыдатнымі для сістэм, дзе патрабуецца наступны аналіз відэашэрагу.

JPEG-2000 прадугледжвае павелічэнне каэфіцыента сціску ў параўнанні з JPEG на 30\%. Гэты метад сціску выкарыстоўвае вeйвлет-пераўтварэнне, дзякуючы чаму характэрныя для JPEG блокавыя скажэнні знікаюць, а каэфіцыент сціску можа дасягаць 200 (хоць пры вялікіх каэфіцыентах сціску з'яўляюцца артэфакты, ствараемыя вeйвлет-пераўтварэннем).
