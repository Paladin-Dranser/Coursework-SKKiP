\sectionWithoutNumber{Заключэнне}

У выніку выканання дадзенага курсавога праекта была спраектавана
сетка правайдара IP-паслуг, якая складаецца з 12 маршрутызатараў,
тэхналогіі абаненцкага доступу LTE. Спраектаваная сетка пабудавана
пры дапамозе паўназвязнай структуры.

Абаненцкая ёмістасць IP-сеткі склала 33200 абанентаў.
Сумарная прапускная здольнасць агрэгіраванага трафіка склала да
3655,36 Мбіт/с, і для яе забеспячэння быў выбраны канал сувязі 10GE,
які не толькі забяспечвае неабходную прапускную здольнасць, аднак
таксама мае дастатковы рэзерв на будучае развіццё сеткі.

Праектаваная сетка пры любой загрузцы (0,2--0,8) не забяспечвае
неабходную якасці прадастаўлення VoIP-паслуг.

Для памяншэння нагрузкі на маршрутызатары М1 і М2 вырашана
выкарыстоўваць палітыкі маршрутызацыі пратакола OSPF, каб
трафік размяркоўваўся на іншыя маршрутызатары, якія
маюць меншую нагрузку і адпавядаюць параметрам добрай якасці
прадастаўлення паслуг.

У праектаванай сетцы выкарыстоўваецца наступнае сеткавае
абсталяванне:
\begin{enumerate}
    \item VoIP шлюз SNR-VG-6040;
    \item IPTV сервер Cisco IPTV 3427-C1;
    \item маршрутызатар NE40E-X8.
\end{enumerate}
