\sectionWithoutNumber{Заключэнне}

У выніку выканання дадзенага курсавога праекта была спраектавана
сетка правайдара IP-паслуг, якая складаецца з 12 маршрутызатараў,
тэхналогіі абаненцкага доступу LTE. Спраектаваная сетка пабудавана
пры дапамозе паўназвязнай структуры.

Абаненцкая ёмістасць IP-сеткі склала 33200 абанентаў.
Сумарная прапускная здольнасць агрэгіраванага трафіка склала да
3655,36 Мбіт/с, і для яе забеспячэння быў выбраны канал сувязі 10GE,
які не толькі забяспечвае неабходную прапускную здольнасць, аднак
таксама мае дастатковы рэзерв на будучае развіццё сеткі.

Праектаваная сетка пры любой загрузцы (0,2--0,8) не забяспечвае
неабходную якасці прадастаўлення VoIP-паслуг.
Так як сеткавыя параметры для VoIP перавышаюць дапушчальныя межы,
а значыць якасць сеткі ва ўмовах агрэгіраванага трафіку не адпавядае
параметрам добрай якасці,
неабходна прыняць дадатковыя меры падчас далейшага планавання
сеткі для паляпшэння сеткавых параметраў.

Для паляпшэння сеткавых параметраў на ўсіх кірунках неабходна
прыняць наступныя меры:
\begin{enumerate}
    \item выкарыстоўваць маршрутызатары з высокай прадукцыйнасцю.
          Асаблівую ўвагу звярнуць на выбар маршрутызатараў
          M1 і M2 (патрабуецца максімальная прадукцыйнасць);
    \item выкарыстоўваць кодэк з меншай хуткасцю;
    \item выкарыстоўваць пратакол маршрутызацыі з гібкай настройкай
          палітык маршрутызацыі, для памяншэння нагрузкі на
          маршрутызатары М1 і М2.
\end{enumerate}

Улічваючы выдатныя сеткавыя параметры для IPTV можна паменшыць
колькасць IPTV каналаў для аптымізацыі кошту пры праектаванні сеткі.

Таксама неабходна разглeдзець магчымасць выкарыстоўвання тэхналогіі
агрэгацыі (Link Aggregation Layer 3) для замены тыпу канала сувязі з
10GE на 4x1GE, так як прапускная здольнасць
10GE (10 Гбіт/с) выкарыстоўваецца
не аптымальна пры атрыманай сумарнай прапускной здольнасці
праектуемай сеткі (3,5 Гбіт/с).

Для памяншэння нагрузкі на маршрутызатары М1 і М2 вырашана
выкарыстоўваць палітыкі маршрутызацыі пратакола OSPF, каб
трафік размяркоўваўся на іншыя маршрутызатары, якія
маюць меншую нагрузку і адпавядаюць параметрам добрай якасці
прадастаўлення паслуг.

У праектаванай сетцы выкарыстоўваецца наступнае сеткавае
абсталяванне:
\begin{enumerate}
    \item VoIP шлюз SNR-VG-6040;
    \item IPTV сервер Cisco IPTV 3427-C1;
    \item маршрутызатар NE40E-X8.
\end{enumerate}

Неабходна адзначыць, што праектаваная сетка прыдатная для
наступнага развіцця паслуг, якія прадастаўляюцца, і павелічэнню
колькасці абанентаў, так як падчас праектавання быў закладзены
запас прапускной здольнасці каналаў (канал сувязі 10GE пры бягучай
загрузцы 3,5 Гбіт/с), абранае сеткавае абсталяванне мае модульную
структуру, што дазваляе лёгка павялічваць прадукцыйнасць абсталявання
заменай модуля на модуль з лепшымі параметрамі.
