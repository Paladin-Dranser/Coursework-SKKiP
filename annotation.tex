\thispagestyle{empty} % анатацыя не нумаруецца
\sectionWithoutContent{Анатацыя}

У дадзеным курсавым праекце разглядаецца праектаванне сеткі правайдара
IP-паслуг ва ўмовах агрэгіраванага трафіка рэальнага часу.

У першым раздзеле разглядаюцца асноўныя зветкі аб IP-сетках, коратка
апісваюцца прынцыпы пабудовы IP-сетак і іх паслуг даны патрабаванні, які
прад'яўляюцца да сетак з трафікам рэальнага часу. Апісаны асноўныя
VoIP і IPTV кодэкі, і іх параметры. Паказана абагульненая структура IP-сеткі.

У другім раздзеле выкананы разлік сеткавых параметраў праектаванай сеткі,
дадзена матэматычная мадэль разліку сеткавых параметраў, зроблен разлік
МІП і разлік канальнага рэсурсу праектаванай сеткі.

У трэцім раздзеле зроблен выбар структуры сеткі і яе элементаў,
разгледжаны асноўныя пратаколы маршрутызацыі. Апісан загаловак пакета
выбранага пратакола маршрутызацыі OSPF, разгледжаны тэхналогіі
пабудовы сеткі абаненцкага доступу. Зроблен выбар сеткавага абсталявання, дадзена апісанне тэхнічных характарыстык.

Аб'ём тлумачальнай запіскі складае 52 старонкі і змяшчае
16 табліц, 6 малюнкаў, 9 фор\-мул, 10 крыніц інфармацыі.
